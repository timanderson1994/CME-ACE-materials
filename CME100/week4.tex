\documentclass[letterpaper, 11pt]{article}
\usepackage{comment} % enables the use of multi-line comments (\ifx \fi) 
\usepackage{lipsum} %This package just generates Lorem Ipsum filler text. 
\usepackage{fullpage} % changes the margin

\usepackage{fancyhdr} % Required for custom headers
\usepackage{lastpage} % Required to determine the last page for the footer
\usepackage{extramarks} % Required for headers and footers
\usepackage{mdframed}
\usepackage{caption}
\usepackage{subcaption}
\usepackage{float}
\usepackage{array}
\usepackage{soul}
\usepackage{amsmath}
\usepackage{graphicx} % Required to insert images
\usepackage{multicol}
\usepackage{enumitem}
\usepackage{amssymb,bm}
\usepackage{verbatim,eufrak,hyperref,bbm}
\usepackage{titlesec}

%%%%% TEMPLATE-SPECIFIC FORMATTING %%%%%
%\usepackage{fourier}
\usepackage[adobe-utopia]{mathdesign}
\titleformat{\section}
  {\normalfont\fontsize{12}{15}\bfseries}{\thesection.}{1em}{}
  \titleformat{\subsection}[runin]{\normalfont}{\thesubsection}{3pt}{}
%\usepackage[T1]{fontenc}

%----------------------------------------------------------------------------------------
%	NAME AND CLASS SECTION
%----------------------------------------------------------------------------------------

\newcommand{\hmwkTitle}{Week\ 4\ Worksheet} % Assignment title
\newcommand{\hmwkClass}{CME\ 100\ ACE} % Course/class
\newcommand{\hmwkAuthorName}{Timothy Anderson} % Your name
\newcommand{\hmwkAuthorEmail}{timmya@stanford.edu} % Your email

% Set up the header and footer
\pagestyle{fancy}
\lhead{} % Top left header
\chead{} % Top center header
\rhead{} % Top right header
\lfoot{\hmwkClass\ : \hmwkTitle} % Bottom left footer
\cfoot{Page\ \thepage\ of\ \pageref{LastPage}} % Bottom center footer
\rfoot{\hmwkAuthorName} % Bottom right footer
\renewcommand\headrulewidth{0pt} % Size of the header rule
\renewcommand\footrulewidth{0.4pt} % Size of the footer rule


% Math commands
\DeclareMathOperator*{\argmin}{arg\,min}
\DeclareMathOperator*{\argmax}{arg\,max}
\allowdisplaybreaks

% Margins
\topmargin=-0.45in
\evensidemargin=0in
\oddsidemargin=0in
\textwidth=6.5in
\textheight=9.0in
\headsep=0.25in 

\setlength{\parindent}{0pt} % Set indent to zero

\begin{document}

%\thispagestyle{empty}
\noindent
\normalsize 
%\hmwkAuthorName 
\hmwkClass \hfill April\ 24,\ 2017\\
%\hmwkAuthorEmail \\

\begin{center} \Large \textbf{\hmwkTitle} \end{center}

\section{Matrix Operations and Linear Systems}
\subsection{} Compute the following using basic matrix operations.
\[
\left[\begin{array}{ccc} 2 & 1 &-1 \end{array} \right]  \left[ \begin{array}{ccc} 1 & 0 & 2 \\ 0 & 3 & 3 \\ 1 & 2 & 1 \end{array}   \right]\left[\begin{array}{c} 2\\ 1 \\-1 \end{array} \right] \qquad \qquad 
\left[\begin{array}{ccc} 0 & 3 & 2 \end{array} \right]  \left[ \begin{array}{ccc} 2 & 1 &0  \\ 4 & 0 & -6 \\ -6 & 0 & 9 \end{array}   \right] \left[\begin{array}{c} 1 \\ 1 \\ -1 \end{array} \right]  
\]

\subsection{} Prove the following:
\begin{enumerate}[label=(\alph*)]
\item Assuming $\bm{A}$ and $\bm{B}$ are invertible, show that $(\bm{A}\bm{B})^{-1} = \bm{B}^{-1} \bm{A}^{-1}$. 

\item Let $ \bm{A} \in \mathbb{R}^{n \times n}$ and $\vec v \in \mathbb{R}^n$. Show that 
\[ (\bm{A} + \vec v \vec v^T )^{-1} = \bm{A} - \frac{1}{1 + \vec v^T \bm{A}^{-1} \vec v} \bm{A}^{-1} \vec v \vec v^T \bm{A}^{-1} \]
\textit{Note:} this is the famous Sherman-Morrison-Woodbury formula, which is an important result in numerical linear algebra. 

\item In linear algebra, we often like to work with what are known as \textit{canonical basis vectors}, which are vectors $\vec e_i$ such that the $i^{th}$ element is 1 and all other elements are 0. Suppose we are given a matrix $\bm{A} \in \mathbb{R}^{n \times n}$ such that for any vector $\vec v \in \mathbb{R}^n$, we have $\vec v ^T \bm{A} \vec v >0$ when $\vec v \neq \bm{0}$. Prove that all of the diagonal entries of $\bm{A}$ are positive and non-zero. 
\par \textit{Note:} these types of matrices are referred to as \textit{positive definite}, and are extremely important in numerical linear algebra and optimization theory. 

\item Suppose I can decompose matrix $\bm{A}$ as $\bm{A} = \bm{Q} \bm{R}$ such that $\bm{Q}^T\bm{Q} = \bm{Q}\bm{Q}^T = \bm{I}$. Show that $(\bm{A}^T\bm{A})^{-1} \bm{A}^T = (\bm{R}^T \bm{R})^{-1} \bm{R}^T \bm{Q}^T$. 

\end{enumerate}

\subsection{Least-squares problems} Suppose I have a set of data points $(x_1,y_2),(x_2,y_2),...,(x_n,y_n)$ and I would like to find the $n^{th}$-degree polynomial
\[y(x) = c_0 + c_1 x + c_2 x^2 + ... + c_n x^n\] 
which exactly passes through these points. Set up a system of equations that you could solve for the $c_i$. 

%\section{Rotation Matrices}


\section{Matrix Inverse \& Gaussian Elimination}
Compute the inverse of the following matrices. If it is not invertible, state why. What is the rank of each matrix? Which matrices will have a unique solution for $\bm{A}x = b$?
\[  \left[ \begin{array}{ccc} 1 & 0 & 2 \\ 0 & 3 & 3 \\ 1 & 2 & 1 \end{array}   \right] \qquad \qquad 
 \left[ \begin{array}{ccc} 1 & 1 & 1 \\ 3 & 2 & 3 \\ 0 & 5 & 5 \end{array}   \right] \qquad \qquad 
  \left[ \begin{array}{ccc} 2 & 4 & -6 \\ 1 & 0 & 0 \\ 0 & -6 & 9 \end{array}   \right] \]

\section{Linear Systems \& Gauss-Jordan Elimination}
Solve the following linear systems using Gauss-Jordan elimination. If there is no solution, state why. 
\[ \left[ \begin{array}{ccc} 1 & 0 & 2 \\ 0 & 3 & 3 \\ 1 & 2 & 1 \end{array}   \right] \mathbf{x} = \left[ \begin{array}{c} 0 \\ 1 \\ 1 \end{array} \right], \qquad \qquad  
\left[ \begin{array}{ccc} 1 & 1 & 2 \\ 0 & 3 & 3 \\ 1 & 2 & 3 \end{array}   \right] \mathbf{x} = \left[ \begin{array}{c} 0 \\ 1 \\ 1 \end{array} \right]\]

\section{Determinants}
\subsection{} Compute the determinant of:
\[  \left[ \begin{array}{ccc} 1 & 0 & 2 \\ 0 & 3 & 3 \\ 1 & 2 & 1 \end{array}   \right] \]
\subsection{} Solve the following system using Cramer's rule. 
\[ \left[ \begin{array}{ccc} 1 & 0 & 2 \\ 0 & 3 & 3 \\ 1 & 2 & 1 \end{array}   \right] \mathbf{x} = \left[ \begin{array}{c} 0 \\ 1 \\ 1 \end{array} \right] \]

\subsection{} If the determinant of $\bm{A} \in \mathbb{R}^{5 \times 5} $ is 3, what is the determinant of $2 \mathbf{A}$?

\subsection{} Does the following system have a unique solution? Why or why not? (Use an answer based on determinants; do not simply restate the result form the previous problem.)
\[ \left[ \begin{array}{ccc} 1 & 1 & 2 \\ 0 & 3 & 3 \\ 1 & 2 & 3 \end{array}   \right] \mathbf{x} = \left[ \begin{array}{c} 0 \\ 1 \\ 1 \end{array} \right] \]



\end{document}