\documentclass[letterpaper, 10pt]{article}
\usepackage{comment} % enables the use of multi-line comments (\ifx \fi) 
\usepackage{lipsum} %This package just generates Lorem Ipsum filler text. 
\usepackage{fullpage} % changes the margin

\usepackage{fancyhdr} % Required for custom headers
\usepackage{lastpage} % Required to determine the last page for the footer
\usepackage{extramarks} % Required for headers and footers
\usepackage{mdframed}
\usepackage{caption}
\usepackage{subcaption}
\usepackage{float}
\usepackage{array}
\usepackage{soul}
\usepackage{amsmath}
\usepackage{mathtools}
\usepackage{graphicx} % Required to insert images
\usepackage{multicol}
\usepackage{enumitem}
\usepackage{amssymb,bm}
\usepackage{verbatim,eufrak,hyperref,bbm}
\usepackage{titlesec}

%%%%% TEMPLATE-SPECIFIC FORMATTING %%%%%
%\usepackage{fourier}
\usepackage[adobe-utopia]{mathdesign}
\titleformat{\section}
  {\normalfont\fontsize{12}{15}\bfseries}{\thesection.}{1em}{}
  \titleformat{\subsection}[runin]{\normalfont}{\thesubsection}{3pt}{}
%\usepackage[T1]{fontenc}

%----------------------------------------------------------------------------------------
%	NAME AND CLASS SECTION
%----------------------------------------------------------------------------------------

\newcommand{\hmwkTitle}{Week\ 2\ Solutions} % Assignment title
\newcommand{\hmwkClass}{CME\ 100\ ACE} % Course/class
\newcommand{\hmwkAuthorName}{Timothy Anderson} % Your name
\newcommand{\hmwkAuthorEmail}{timmya@stanford.edu} % Your email

% Set up the header and footer
\pagestyle{fancy}
\lhead{} % Top left header
\chead{} % Top center header
\rhead{} % Top right header
\lfoot{\hmwkClass\ : \hmwkTitle} % Bottom left footer
\cfoot{Page\ \thepage\ of\ \pageref{LastPage}} % Bottom center footer
\rfoot{\hmwkAuthorName} % Bottom right footer
\renewcommand\headrulewidth{0pt} % Size of the header rule
\renewcommand\footrulewidth{0.4pt} % Size of the footer rule


% Math commands
\DeclareMathOperator*{\argmin}{arg\,min}
\DeclareMathOperator*{\argmax}{arg\,max}
\DeclareMathOperator*{\proj}{proj}
\allowdisplaybreaks

% Margins
\topmargin=-0.45in
\evensidemargin=0in
\oddsidemargin=0in
\textwidth=6.5in
\textheight=9.0in
\headsep=0.25in 

\setlength{\parindent}{0pt} % Set indent to zero
\setlength{\parskip}{5pt}

\begin{document}
%Header-Make sure you update this information!!!!


%\thispagestyle{empty}
\noindent
\normalsize 
%\hmwkAuthorName 
\hmwkClass \hfill April\ 10,\ 2017\\
%\hmwkAuthorEmail \\

\begin{center} \Large \textbf{\hmwkTitle} \end{center}


\section{Unit Vectors}
\subsection{} Describe conceptually what a unit vector is. 
\par \textbf{Solution:} A unit vector is a vector with unit magnitude. On a conceptual level, this is the vector equivalent of the number 1, since we can scale a unit vector by a scalar amount to get any vector in that direction. The simplest unit vectors are the $\bm{\vec i}$, $\bm{\vec j}$, and $\bm{\vec k}$ vectors we work with in Cartesian space, but in fact we could use any three \textit{orthogonal} unit vectors in cartesian space to use to decompose a 3-dimensional vector into separate components. 
\par As an aside, it is obvious that we can decompose any 3-dimensional vector into the $\bm{\vec i}$, $\bm{\vec j}$, and $\bm{\vec k}$ vectors. This means that these three vectors form a \textit{basis} for our space. However, this is not the only basis. Indeed, any three mutually orthogonal unit vectors could be used as a basis for 3-dimensional cartesian space. (You can think of any set of vectors like this being a rotation of the three we typically work with.) Because the $\bm{\vec i}$, $\bm{\vec j}$, and $\bm{\vec k}$ vectors only have one non-zero component in each direction, we refer to these as our \textit{canonical basis} vectors. 

\subsection{} Compute the unit vectors for the following vectors.
\begin{enumerate}[label=(\alph*)]
\item $\vec v = \langle 1,1,1 \rangle$
\par \textbf{Solution:} Remember that the unit vector is simply the original vector \textit{normalized} by its magnitude. Therefore:
\[ \frac{\vec v}{|\vec v|} = \frac{1}{\sqrt{1^2 + 1^2 + 1^2}} \vec v = \frac{\sqrt{3}}{3} \vec v \quad\blacksquare \]

\item $\vec w = \langle 0,-1,-1 \rangle$
\par \textbf{Solution:} 
\[ \frac{ \vec w}{|\vec w|} = \frac{1}{\sqrt{ 0^2 + (-1)^2 + (-1)^2}} \vec w = \frac{\sqrt{2}}{2} \vec w \]

\item $\vec u = \langle 10,8,-7 \rangle$
\par \textbf{Solution:} 
\[ \frac{\vec u}{|\vec u|} = \frac{1}{\sqrt{10^2 + 8^2 + (-7)^2}} \vec u = \frac{\sqrt{213}}{213} \vec u \]

\end{enumerate}

\subsection{} The unit vector from point $A = (0,2,3)$ to $B = (1,6,-2)$.
\par \textbf{Solution:} First find the vector between the two points, calculate the magnitude, then find the unit vector:
\begin{gather*}
B - A = \langle 1 - 0, 6-2, -2-3 \rangle = \langle 1, 4, -5 \rangle \\
|B - A| = \sqrt{1^2 + 4^2 + (-5)^2} = \sqrt{42} \\
\frac{B - A}{|B - A|} = \frac{\sqrt{42}}{42}\langle 1, 4, -5 \rangle \quad\blacksquare
\end{gather*}

\subsection{} If $\vec v$ and $\vec w$ are orthogonal, will their unit vectors also be orthogonal? Why? 
\par \textbf{Solution:} Yes, their associated unit vectors are just $\vec v$ and $\vec w$ scaled by their magnitudes. So the dot product of the unit vector will still be zero:
\[ \vec v \cdot \vec w = 0 \implies \left( \frac{\vec v}{|\vec v|}  \right) \cdot \left( \frac{\vec w}{|\vec w|}  \right) = \frac{1}{|\vec v||\vec w|}(\vec v \cdot \vec w) = 0 \quad\blacksquare \]

\section{Vector Operations}
Compute the following.
\begin{enumerate}[label=(\alph*)]
\item[] \[ \vec v = \langle 1,2,5 \rangle, \quad \vec w = \langle 3, -4, 2 \rangle \]

\item $2 \vec v  - \vec w$
\par \textbf{Solution:}
\[ 2 \vec v  - \vec w = 2 \langle 1,2,5 \rangle - \langle 3, -4, 2 \rangle = \langle 2 - 3,4 - (-4),10 - 2 \rangle = \langle -1,8,8 \rangle \quad\blacksquare \]

\item $(2\vec v) \cdot \vec w$
\par \textbf{Solution:}
\[ (2 \vec v)  \cdot \vec w = (2 \langle 1,2,5 \rangle) \cdot( \langle 3, -4, 2 \rangle) = 2 \cdot 3 + 4 \cdot (-4) + 10 \cdot 2 = 42 \quad\blacksquare \]

\item The unit vector of $\vec v \times \vec w$
\par \textbf{Solution:} first find the cross product, then the magnitude, and finally the unit vector:
\begin{gather*}
\vec v \times \vec w = (2 \cdot 2 - (-4)\cdot 5) \vec i - (1 \cdot 2 - 3 \cdot 5) \vec j + (1 \cdot (-4) - 3 \cdot 2) \vec k  = 24 \vec i + 13 \vec j - 10 \vec k \\
|\vec v \times \vec w| = \sqrt{24^2 + 13^2 + (-10)^2} = \sqrt{845} = 13\sqrt{5} \\
\frac{\vec v \times \vec w}{|\vec v \times \vec w|} = \frac{\sqrt{5}}{65} \langle 24,13,-10 \rangle\quad \blacksquare
\end{gather*}


\item[] \[ \vec v = \langle 1,0,0 \rangle, \quad \vec w = \langle \sqrt{3}, \sqrt{3},\sqrt{3} \rangle \]

\item $\vec v \cdot \left( \frac{1}{\sqrt{3}} \vec w \right)$
\par \textbf{Solution:}
\[\vec v \cdot \left( \frac{1}{\sqrt{3}} \vec w \right) = \langle 1,0,0 \rangle \cdot \left( \frac{1}{\sqrt{3}} \langle \sqrt{3}, \sqrt{3},\sqrt{3} \rangle \right) = 1 \cdot 1 + 0 \cdot 1 + 0 \cdot 1 = 1 \quad\blacksquare \]


\item $\vec w \cdot \vec w$
\par \textbf{Solution:}
\[ \vec w \cdot \vec w = \langle \sqrt{3}, \sqrt{3},\sqrt{3} \rangle \cdot \langle \sqrt{3}, \sqrt{3},\sqrt{3} \rangle = 3(\sqrt{3}^2) = 27 \quad\blacksquare \]

\item The angle between $\vec v$ and $\vec w$. 
\par \textbf{Solution:} The easiest formula is use is $\theta = \cos^{-1} \left( \frac{ \vec v \cdot \vec w}{|\vec v||\vec w|} \right)$.
\[ \theta = \cos^{-1} \left( \frac{ \vec v \cdot \vec w}{|\vec v||\vec w|} \right) = \cos^{-1} \left( \frac{ \sqrt{3}}{|1||3\sqrt{3}|} \right) | \approx 54.74^\circ\]


\end{enumerate}

\section{Projections}
Compute the following.
\begin{enumerate}[label=(\alph*)]
\item[] \[ \vec v = \langle 1,2,5 \rangle, \quad \vec w = \langle 3, -4, 2 \rangle \]

\item $\proj_{\vec v} \vec w$
\par \textbf{Solution:}
\[ \proj_{\vec v} \vec w = \frac{ \vec v \cdot \vec w}{|\vec v|} \frac{ \vec v }{|\vec v|} =  
\frac{ 3 - 8 + 10}{\sqrt{1^2 + 2^2 + 5^2}} \frac{1}{\sqrt{1^2 + 2^2 + 5^2}}\langle 1,2,5 \rangle = 
\frac{ 5}{\sqrt{30}} \frac{1}{\sqrt{30}}\langle 1,2,5 \rangle  = \frac{1}{6}\langle 1,2,5 \rangle \quad\blacksquare
\]

\item $\proj_{\vec w} \vec v$
\par \textbf{Solution:}
\[ \proj_{\vec w} \vec v = \frac{ \vec v \cdot \vec w}{|\vec w|} \frac{ \vec w }{|\vec w|} =  
\frac{ 3 - 8 + 10}{\sqrt{3^2 + (-4)^2 + 2^2}} \frac{1}{\sqrt{3^2 + (-4)^2 + 2^2}}\langle 3, -4, 2 \rangle = 
\frac{ 5}{\sqrt{29}} \frac{1}{\sqrt{29}}\langle 3, -4, 2 \rangle  = \frac{5}{29}\langle 3, -4, 2 \rangle \quad\blacksquare
\]

\item[] \[ \vec v = \langle 1,0,0 \rangle, \quad \vec w = \langle \sqrt{3}, \sqrt{3},\sqrt{3} \rangle \]

\item $\proj_{\vec v} \vec w$
\par \textbf{Solution:}
\[ \proj_{\vec v} \vec w = \frac{ \vec v \cdot \vec w}{|\vec v|} \frac{ \vec v }{|\vec v|} =  
\frac{ \sqrt{3} +0 + 0}{\sqrt{1^2 + 0^2 + 0^2}} \frac{1}{\sqrt{1^2 + 0^2 + 0^2}}\langle 1,0,0 \rangle = 
\frac{ \sqrt{3}}{1}\langle 1,0,0 \rangle  = \langle \sqrt{3},0,0 \rangle \quad\blacksquare
\]

\item $\proj_{\vec w} \vec v$
\par \textbf{Solution:}
\[ \proj_{\vec w} \vec v = \frac{ \vec v \cdot \vec w}{|\vec w|} \frac{ \vec w }{|\vec w|} =  
\frac{ \sqrt{3} +0 + 0}{\sqrt{\sqrt{3}^2 + \sqrt{3}^2 + \sqrt{3}^2}} \frac{1}{\sqrt{\sqrt{3}^2 + \sqrt{3}^2 + \sqrt{3}^2}} \langle \sqrt{3}, \sqrt{3}, \sqrt{3} \rangle = 
\frac{ \sqrt{3} }{3} \frac{1}{3} \langle \sqrt{3}, \sqrt{3}, \sqrt{3} \rangle  = \frac{1}{3}\langle 1, 1, 1 \rangle \quad\blacksquare
\]

\end{enumerate}

\section{Lines and Planes}
\subsection{} Compute the area enclosed by the parallelogram defined by:
\[ A(0,0) \quad B(7,3) \quad C(9,8) \quad D(2,5) \]
\par \textbf{Solution:} Recall that the area of a parallelogram is just the magnitude of the cross product of two of its sides. Here we will pick $AD$ and $AB$ as the sides and find $|AB \times AD|$. 
\begin{gather*}
AB = \langle 7,3 \rangle, \quad AD = \langle 2,5 \rangle \\
AB \times AD = \langle 0, 0,  35 - 6\rangle = \langle 0, 0,  29\rangle \\
|AB \times AD| = 29 \quad\blacksquare
\end{gather*}

\subsection{} Compute the area enclosed by the triangle defined by:
\[ A(0, 0)\quad B(-2,3)\quad C(3, 1) \]
\par \textbf{Solution:} Similarly, recall that the area of a triangle is half the magnitude of the cross product of two of its sides. Here we will pick $AB$ and $AC$ as the sides and find $|AC \times AB|$. 
\begin{gather*}
AC = \langle 3,1 \rangle, \quad AB = \langle -2,3 \rangle \\
AB \times AD = \langle 0, 0,  9 - (-2)\rangle = \langle 0, 0,  11\rangle \\
\frac{1}{2} |AB \times AD| = \frac{11}{2} \quad\blacksquare
\end{gather*}

\subsection{} Find the equation for the line through $(1,2,1)$ in the direction of $\vec v = \langle 0, 1, 0 \rangle$
\par \textbf{Solution:} we parameterize the line by $ t \in \mathbb{R}$ and construct the line $L = P_0 + t \vec v$.
\[ L = P_0 + t \vec v = \langle 1,2,1 \rangle + t \langle 0,1,0 \rangle = \langle 1,2  + t,1 \rangle \quad\blacksquare \]

\subsection{} Find the equation for the plane through $(1,2,1)$ with normal $\vec n = \langle -1, 0, 1 \rangle$
\par \textbf{Solution:} a plane is just a set of points, and we are looking for an equation that defines the set of points $(x,y,z)$ such that $\vec n$ is always orthogonal to a vector formed between these points and our points $P_0 = (1,2,1)$. From this, we can easily re-derive the plane equation:
\[  0 = \vec n \cdot (\langle x,y,z \rangle - \langle 1,2,1 \rangle) = \langle -1, 0, 1 \rangle \cdot \langle x-1,y-2,z-1 \rangle = -(x-1) + (z-1) = 0 \quad\blacksquare \]


\section{Vector-valued functions}
Compute the velocity and acceleration vectors of the following. In (c), also compute the tangent vector at the given point.
\begin{enumerate}[label = (\alph*)]
\item $\vec r (t) = (1 + t) \vec i + \frac{t^2}{\sqrt{2}} \vec j + \frac{t^3}{3} \vec k$
\par \textbf{Solution:}
\begin{gather*}
\vec v(t) = \vec i + \sqrt{2}t \vec j + t^2 \vec k \quad\blacksquare \\
\vec a(t) = \sqrt{2} \vec j + 2t \vec k \quad\blacksquare
\end{gather*}


\item $\vec r(t) = \sec(t) \vec i + \tan (t) \vec j + t\vec k$
\par \textbf{Solution:}
\begin{gather*}
\vec v(t) =  \tan(x)\sec(x) \vec i + \sec(x)^2 \vec j + \vec k \quad\blacksquare \\
\vec a(t) = (\sec(x)\tan(x)^2 + \sec(x)^3) \vec i + 2 \tan(x) \sec(x)^2 \vec j  \quad\blacksquare  \\
\end{gather*}

\item $\vec r(t) =  \ln(t) \vec i+\frac{t - 1}{t + 2} \vec j +t\ln(t) \vec k$, and $t_0 = 1$
\par \textbf{Solution:}
\begin{gather*}
\vec v(t) = \frac{1}{t} \vec i + \frac{3}{(t + 2)^2} \vec j +  (\ln(t) + 1) \vec k \quad\blacksquare \\
\vec a(t) = \frac{-1}{t^2} \vec i - \frac{6}{(t + 2)^3} \vec j + \frac{1}{t} \vec k \quad\blacksquare 
\end{gather*}

\end{enumerate}


%\section{Partial Derivatives}
%\subsection{} Compute the following partial derivatives.
%\begin{enumerate}[label=(\alph*)]
%\item
%
%
%\item
%
%
%\item
%
%
%\item
%
%
%\end{enumerate}
%
%
%\subsection{Application} Suppose we have a vector $\vec \Delta$ such that 
%\[ \vec \Delta = \left[ \begin{array}{c} \frac{\partial}{\partial x} \\ \frac{\partial}{\partial y} \\ \frac{\partial}{\partial z} \end{array} \right] \]
%That is, $\vec \Delta $ when multiplied by a function will produce a vector of the infinitesimal change in that function in each direction at a given point $(x,y,z)$. We call this the \textit{gradient vector}, which you will learn about later in this course. (Why would something like this be useful?)


\end{document}