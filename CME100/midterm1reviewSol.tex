	\documentclass[letterpaper, 11pt]{article}
\usepackage{comment} % enables the use of multi-line comments (\ifx \fi) 
\usepackage{lipsum} %This package just generates Lorem Ipsum filler text. 
\usepackage{fullpage} % changes the margin

\usepackage{fancyhdr} % Required for custom headers
\usepackage{lastpage} % Required to determine the last page for the footer
\usepackage{extramarks} % Required for headers and footers
\usepackage{mdframed}
\usepackage{caption}
\usepackage{subcaption}
\usepackage{float}
\usepackage{array}
\usepackage{soul}
\usepackage{amsmath}
\usepackage{graphicx} % Required to insert images
\usepackage{multicol}
\usepackage{enumitem}
\usepackage{amssymb,bm}
\usepackage{verbatim,eufrak,hyperref,bbm}
\usepackage{titlesec}
\usepackage{listings}

%%%%% TEMPLATE-SPECIFIC FORMATTING %%%%%
%\usepackage{fourier}
\usepackage[adobe-utopia]{mathdesign}
\titleformat{\section}
  {\normalfont\fontsize{12}{15}\bfseries}{\thesection.}{1em}{}
  \titleformat{\subsection}[runin]{\normalfont}{\thesubsection}{3pt}{}
%\usepackage[T1]{fontenc}

%----------------------------------------------------------------------------------------
%	NAME AND CLASS SECTION
%----------------------------------------------------------------------------------------

\newcommand{\hmwkTitle}{Midterm\ 1\ Review\ Solutions} % Assignment title
\newcommand{\hmwkClass}{CME\ 100\ ACE} % Course/class
\newcommand{\hmwkAuthorName}{T Anderson, S Messingher, A Kusimo} % Your name
\newcommand{\hmwkAuthorEmail}{timmya@stanford.edu} % Your email

% Set up the header and footer
\pagestyle{fancy}
\lhead{} % Top left header
\chead{} % Top center header
\rhead{} % Top right header
\lfoot{\hmwkClass\ : \hmwkTitle} % Bottom left footer
\cfoot{Page\ \thepage\ of\ \pageref{LastPage}} % Bottom center footer
\rfoot{\hmwkAuthorName} % Bottom right footer
\renewcommand\headrulewidth{0pt} % Size of the header rule
\renewcommand\footrulewidth{0.4pt} % Size of the footer rule


% Math commands
\DeclareMathOperator*{\argmin}{arg\,min}
\DeclareMathOperator*{\argmax}{arg\,max}
\allowdisplaybreaks

% Margins
\topmargin=-0.45in
\evensidemargin=0in
\oddsidemargin=0in
\textwidth=6.5in
\textheight=9.0in
\headsep=0.25in 

\setlength{\parindent}{0pt} % Set indent to zero

\begin{document}

%\thispagestyle{empty}
\noindent
\normalsize 
%\hmwkAuthorName 
\hmwkClass \hfill April\ 23,\ 2017\\
%\hmwkAuthorEmail \\

\begin{center} \Large \textbf{\hmwkTitle} \end{center}

\section{The Algebra of Vectors}

\section{Dot Products}

\section{Cross Products}

\section{Lines and Planes}

\section{Vector-Valued Functions}

\section{Velocity and Acceleration Vectors}

\begin{enumerate}
\item Given
\begin{align*}
\vec{\bm{r}}(t)= (4\cos t)\vec{\bm{i}}+(4\sin t)\vec{\bm{j}}+3t\vec{\bm{k}}
\label{eq:traj1}
\end{align*}

find the arc length from $(4,0,0)$ to $(0,4,\frac{3}{2}\pi)$ (Problem 11 from Section 13.3 of textbook)

\par \textbf{Solution:} To get arc length, we need to integrate $\vec{\bm{|v|}}$ with respect to time. 

But what are the boundaries of integration? Note that the given boundaries are in terms of spatial coordiantes, not time values. We need to convert these limits to time. 

What does the starting point (4,0,0) correspond to in time? Well, we know that the component of the trajectory along $\vec{\bm{k}}$ can only be zero at $t=0$. The $\vec{\bm{i}}$ and $\vec{\bm{j}}$ coordinates at $t=0$ are 4 and 0, respectively. This complies with the initial point given. Using same logic, second boundary point $(0,4,\frac{3}{2}\pi)$ corresponds to $t=\pi/2$.

We can then find
\begin{gather*}
\vec{\bm{r'}}(t)= (-4\sin t)\vec{\bm{i}} + (4\cos t)\vec{\bm{j}} + 3\vec{\bm{k}}\\
\vec{\bm{|v|}}= \sqrt{4^2(\sin ^2 t\vec{\bm{i}}+\cos ^2 t)+3^2}=5\\
L =  5 \int_{0}^{\pi/2}dt=2.5\pi\quad\blacksquare
\end{gather*}
\item Given the parameterized trajectory 
\[ \vec{\bm{r}}(t)= (\cos t)\vec{\bm{i}}+(\sin t)\vec{\bm{j}}+(1-\cos t)\vec{\bm{k}}\]

determine whether the acceleration of the trajectory is always parallel to the plane described by $x+z=1$ (similar Problem 17.c from Section 13.3 of textbook) 

\par \textbf{Solution:} Think of this problem as the trajectory of a particle. Let's first find the acceleration vector
\begin{gather*}
\vec{\bm{r'}}(t)= -\sin t\vec{\bm{i}}+\cos t\vec{\bm{j}}+\sin t\vec{\bm{k}} \\
\vec{\bm{a}}=\vec{\bm{r''}}(t)= \cos t\vec{\bm{i}}-\sin t\vec{\bm{j}}+\cos t\vec{\bm{k}}
\end{gather*}

If $\vec{\bm{a}}$ always is parallel to the given plane, we know $\vec{\bm{a}}$ must be orthogonal (perpendicular) to a normal vector $\vec{\bm{n}}$ of that plane. In other words, if we find $\vec{\bm{n}}$, dot it with $\vec{\bm{a}}$, and get 0 as a result, we know $\vec{\bm{a}}$ is always parallel to the plane. \textit{Make sure you understand this logic up to this point before proceeding.}

Now, let's find a vector normal to the plane. To do this, let's define $\vec{\bm{P_0}}=(x_0,y_0,z_0)$ to be a point on the plane. Let's also define $\vec{\bm{P}}=(x,y,z)$ any point on the plane. Then $\vec{\bm{P}}-\vec{\bm{P_0}}$ must be a vector that lines on the plane. 

We want $\vec{\bm{n}}=(n_x,n_y,n_z)$ such that 
\begin{gather*}
(\vec{\bm{P}}-\vec{\bm{P_0}})\cdot \vec{\bm{n}}=0 \\ 
\Rightarrow (x-x_0)n_x+(y-y_0)n_y+(z-z_0)n_z=0 \\
\Rightarrow xn_x+yn_y+zn_z = (x_0n_x+y_0n_y+z_0n_Z)
\end{gather*}

but from the plane equation, we know $x + 0 + z = 1$. From these two equations, we can find $n_x = 1$, $n_z = 1$, and $n_y = 0$. 
\par So, $\vec{\bm{n}}=(1,0,1)$ \textit{this vector is not normalized, and there is no need for it to be normalized for our case.}
\par Finally, $\vec{\bm{a}}\cdot\vec{\bm{n}}=-\cos t + \cos t = 0$, meaning  $\vec{\bm{a}}$ is always parallel to the plane.$\quad\blacksquare$ 

\end{enumerate}

\section{$\vec T$, $\vec N$, and $\vec B$ Vectors, Curvature, and Torsion}

\begin{enumerate}

\item Find $\vec{\bm{T}}$, $\vec{\bm{N}}$, and $\bm{\kappa}$ (curvature) for the space curve
$$\vec{\bm{r}}(t)= (e^t\cos t)\vec{\bm{i}}+(e^t \sin t)\vec{\bm{j}}+2\vec{\bm{k}}$$ 
(Problem 11 from Section 13.4 of textbook)

\par \textbf{Solution:} 
\par It is given that
\[ \vec r(t) = e^t \cos t \bm{i} + e^t \sin t \bm{j} + 2 \bm{k}\]
so we can differentiate once with respect to $t$ to find the velocity:
\begin{align*}
\vec v &= \left( e^t \cos t - e^t \sin t\right) \bm{i} + \left(e^t \sin t + e^t \cos t \right) \bm{j} + 0 \bm{k} \\
|\vec v | &= \sqrt{\left( e^t \cos t - e^t \sin t\right)^t + \left(e^t \sin t + e^t \cos t \right)^t } \\
&= \sqrt{e^{2t}\left( \cos^2 t + \sin^2 t - 2 \sin t \cos t + \sin^2 t + \cos^2 t + 2 \sin t\cos t\right)}\\
&= e^t \sqrt{2} 
\end{align*}
Using this, we can calculate the unit tangent vector $vec T$:
\begin{align*}
\vec T  &= \frac{ \vec v}{|\vec v|}\\
&= \frac{1}{e^t \sqrt{2}} \left( \left( e^t \cos t - e^t \sin t\right) \bm{i} + \left(e^t \sin t + e^t \cos t \right) \bm{j}\right) \\
&= \frac{1}{\sqrt{2}}\left( \cos t - \sin t\right) \bm{i} +  \frac{1}{\sqrt{2}}\left(\sin t + \cos t \right) \bm{j} \quad\blacksquare
\end{align*}
From here, we can differentiate with respect to $t$ to find the unit normal vector $\vec N$:
\begin{align*}
\frac{d\vec T}{dt} &= \frac{1}{\sqrt{2}}\left( -\sin t - \cos t\right) \bm{i} +  \frac{1}{\sqrt{2}}\left(\cos t -\sin t \right) \bm{j} \\
\left| \frac{d \vec T}{dt} \right| &= \sqrt{\left( \frac{1}{\sqrt{2}}\left( -\sin t - \cos t\right)\right)^2 + \left(\frac{1}{\sqrt{2}}\left(\cos t -\sin t \right)\right)^2 } \\
&= \frac{1}{\sqrt{2}} \sqrt{2} = 1\\
\vec N &= \frac{1}{|d \vec T/dt|} \frac{d\vec T}{dt} \\
&= -\frac{1}{\sqrt{2}}\left( \sin t  \cos t\right) \bm{i} +  \frac{1}{\sqrt{2}}\left(\cos t -\sin t \right) \bm{j} \quad\blacksquare
\end{align*}
Finally, we can calculate curvature:
\begin{align*}
\kappa &= \frac{1}{|\vec v|} \left | \frac{d\vec T}{dt} \right| \\
&= \frac{1}{e^t \sqrt{2}} \quad\blacksquare
\end{align*}
\textit{Note:} for a problem such as this, to find the curvature, you effectively need to compute all of the information you would need to find the TNB frame vectors \textit{except} for the binormal vector. This is different from torsion, where we only need to compute the first three derivatives of the path then plug into the determinant formula. 

% \clearpage
% \begin{figure}[H]
% \includegraphics[width=0.6\textwidth]{curvature_sols.jpg}
% \centering
% \end{figure}

\item For the same trajectory, find $\vec{\bm{B}}$,  and $\bm{\tau}$ (torsion)  

(Problem 11 from Section 13.4 of textbook) 

Hint: thinking of torsion intuitively might save you a bunch of time on this one...

\par \textbf{Solution:} To find the binormal vector, recall the formula
\[ \vec B = \vec T \times \vec N \]
which we find as:
\begin{align*} 
\vec T \times \vec N &= \left| \begin{array}{ccc} \vec i & \vec j & \vec k \\
\frac{1}{\sqrt{2}}(\cos t - \sin t) & \frac{1}{\sqrt{2}}(\cos t - \sin t ) & 0 \\
\frac{1}{\sqrt{2}}(-\cos t- \sin t) & \frac{1}{\sqrt{2}}(\cos t - \sin t) & 0 \end{array} \right| \\
&= 0 \vec i + 0 \vec j + \left( \frac{1}{2}(\cos t - \sin )^2 + \frac{1}{2} ( \cos t + \sin t )^2 \right) \vec k \\
&= \vec k \quad\blacksquare
\end{align*}

% \begin{figure}[H]
% \includegraphics[width=0.6\textwidth]{finding_binormal.jpg}
% \centering
% \end{figure}

To find torsion, we need to find the first three derivatives of the path:
\begin{align*}
\vec r(t) &= e^t \cos t \vec i + e^t \sin t \vec j + 2 \vec k \\
\vec v(t) &= r'(t) = \left( e^t \cos t - e^t \sin t\right) \vec i + \left(e^t \sin t + e^t \cos t \right) \vec j + 0 \vec k \\
\vec a(t) &= r''(t) = -2e^t \sin t \vec i + 2 e^t \cos t \vec j + 0 \vec k \\
r'''(t) &= -2e^t ( \sin t + \cos t) \vec i + 2 e^t ( \cos t - \sin t) \vec j + 0 \vec k 
\end{align*}

% \begin{figure}[H]
% \includegraphics[width=0.6\textwidth]{finding_torsion.jpg}
% \centering
% \end{figure}
From here, we can compute $|\vec v \times \vec a|$:
\begin{align*}
\vec v \times \vec a &= \left| \begin{array}{ccc} \vec i & \vec j & \vec k \\
e^t(\cos t - \sin t) & e^t (\cos t + \sin t) & 0 \\
-2e^t \sin t & 2e^t \cos t & 0 \end{array} \right|\\
&= 2e^{2t} \vec k \\
|\vec v \times \vec a | &= 2 e^{2t}
\end{align*}
Recall the expression for torsion. \textit{Note:} there exist multiple expressions for torsion, but this is the expression you will almost always use to actually compute a specific value.
\[ \tau = \frac{ \left| \begin{array}{ccc} \dot x & \dot y & \dot z \\
\ddot x & \ddot y & \ddot z  \\ \dddot x &\dddot y &\dddot z \end{array} \right| }{ |\vec v \times \vec a|^2} = 
\frac{ \left| \begin{array}{ccc} e^t(\cos t - \sin t) & e^t(\cos t + \sin t) & 0 \\
-2e^t \sin t & 2e^t \sin t & 0  \\ -2e^t(\sin t + \cos t) &2e^t(\cos t - \sin t) & 0  \end{array} \right| }{ 4e^{4t}}
= 0 \quad\blacksquare\]
(The determinant here we immediately know is 0 since the entire third column is 0.) 
\par Note that just by looking at the curve, we can see that it lies on a plane parallel to the xy plane that cuts the z axis at 2. This means that the trajectory lives in a plane and it never leaves. This implies that it's torsion \textit{must} be zero, since torsion is an inherently three dimensional phenomenon.

\end{enumerate}

\section{Matrix Operations}
\begin{enumerate}
\item Assuming that $\bm{A}$ and $\bm{A}^T$ is invertible, prove that 
\[ (\bm{A}^T)^{-1} = (\bm{A}^{-1})^T \]
\par \textbf{Solution:} Remember that by definition, the inverse of matrix $\bm{M}$ will satisfy:
\[ \bm{M}\bm{M}^{-1} = \bm{M}^{-1} \bm{M} = \bm{I} \]
Using that $\bm{A}$ is invertible:
\begin{align*}
\bm{I} &= \bm{A} \bm{A}^{-1} \\
&= \left( \bm{A}^{-1}\right)^T \bm{A}^T \\
\intertext{Multiply from the right by $\left(\bm{A}^T\right)^{-1}$:}
\left( \bm{A}^T\right)^{-1} &= \left( \bm{A}^{-1}\right)^T \quad\blacksquare
\end{align*}


\item Compute the inverse of the following matrix:
\[ \bm{A} = \left[ \begin{array}{cc} 3 & 10 \\ 3 & 3 \end{array} \right] \]
\par \textbf{Solution:} Recall the formula for the inverse of a $2 \times 2$ matrix:
\[ \bm{A} = \left[ \begin{array}{cc} a & b \\ c & d \end{array} \right] \qquad \bm{A}^{-1} =\frac{1}{\det\bm{A}} \left[\begin{array}{cc} d & -c \\ -b & a \end{array} \right] \]
Using this, we can easily compute the inverse of the given matrix:
\[ \bm{A}^{-1} = \frac{1}{3(3) - 3(10)} \left[ \begin{array}{cc} 3 & -10 \\ -3 & 3 \end{array} \right] = \left[ \begin{array}{cc} - \frac{1}{7} & \frac{10}{21} \\ \frac{1}{7} & -\frac{1}{7}\end{array} \right] \]

\item Compute the determinant of the following matrix:
\[ \bm{A}  = \left[ \begin{array}{cccc} 4 & 3 & 2 & 1 \\ 3 & 4 & 3 & 2 \\ 2 & 3 & 4 & 3 \\ 1 & 2 & 3 &4 \end{array} \right] \]
\par \textbf{Solution:} The shortcut methods we have previously discussed in section will not work for anything larger than a $3\times 3$ matrix, so we instead will need to use the definition based on principle minors to compute the determinant:
\begin{align*} 
\det \bm{A} &= 4 \det \left| \begin{array}{ccc} 4 & 3 & 2 \\ 3 & 4 &3 \\ 2 &3 &4 \end{array} \right| - 3\det \left| \begin{array}{ccc} 3 & 3 & 2 \\ 2 & 4 &3 \\ 1 & 3 & 4 \end{array} \right| + 2 \det \left| \begin{array}{ccc} 3 & 4 & 2 \\ 2 & 3 & 3 \\ 1 & 2 & 4 \end{array} \right| - \det \left| \begin{array}{ccc} 3 & 4 & 3 \\ 2 & 3 & 4 \\ 1 & 2 & 3 \end{array} \right|
\intertext{From here, compute the determinant of the $3 \times 3$ matrices using the shortcut method.}
%&= 4 (4^3 + 3^2(2) + 3^2(3) - 2^2(4) - 3^2(4) - 3^2(4) ) - 3((3)4^2 + 3^2(1) + 2^2(3) - 4(1)(2) - 3^3 - 4(2)(3)) + \\
%&\qquad 2 ( 3^2(4) + 4(3)(1) + 2^3 - 1(3)(2) - (2)3^2 - (2)4^2 ) - (3^3 + 4^2(1) + (3)2^2 - (1)3^2 - 2(4)(3) - 3(2)(4) ) \\
&= 4 (64 + 18 + 18 - 16 -36 - 36 ) - 3(48 + 9 + 12 - 8 - 27 - 24) + \\
&\qquad 2 ( 36 +12 + 8 - 6 - 18 - 32 ) - (27 + 16 + 12 - 9 - 24 - 24 ) \\
&= 4(12) - 3(10) + 2(0) - (-2) \\
&= 20 \quad\blacksquare
\end{align*}
\end{enumerate}


\section{MATLAB}
\begin{enumerate}
\item The \texttt{A = diag(v)} function in MATLAB takes a vector \texttt{v} and outputs a matrix \texttt{A} such that \texttt{v} is on the diagonal. In MATLAB, we often want to \textit{vectorize} our computations such that we avoid loops and write our code as matrix multiplications, so a function such as \texttt{diag()} is extremely useful. 
\par Suppose we have a column vector $v$, and we want to build a matrix $B$ with $n$ rows such that $v^T$ is every row of the matrix. Write a piece of MATLAB code that does this using
\begin{enumerate}[label=(\alph*)]
\item Two loops
\par \textbf{Solution:}
\begin{lstlisting}[language=Octave]
B = zeros(n,numel(v));
for i = 1:numel(v)
	for j = 1:n
		B(j,i) = v(i);
	end
end
\end{lstlisting}

\item One loop
\par \textbf{Solution:}
\begin{lstlisting}[language=Octave]
for j = 1:n
	B(j,:) = v';
end
\end{lstlisting}

\item No loops
\par \textbf{Solution:}
\begin{lstlisting}[language=Octave]
B = ones(n,numel(v))*diag(v);
\end{lstlisting}

\end{enumerate}
Assume that \texttt{v} and \texttt{n} have already been declared and are stored in MATLAB's memory.
 
\item Recall the interpolation problem discussed in the week 4 ACE worksheet. In this problem, we are given data points $(x_1,y_1), \; (x_2, y_2),\; ... \;,\; (x_n, y_n)$, and want to fit an $n^{th}$ degree polynomial such that:
\[c_0 + c_1x_i + c_2x_i^2 +\cdots + c_nx_i^n = y_i \]
for all $i = 1, ..., n$. 
\begin{enumerate}[label = (\alph*)]
\item Write this as a matrix equation. (This was the same question as in the linear algebra review sheet.)
\par \textbf{Solution:} We know that at every $x_k$, we need to satisfy
\[ y_k = c_0 + c_1 x_k + c_2 x_k^2 + ... + c_n x_k^n \]
So, we can set up this equation for all $i$, which will give us a system of equations we can solve for the $c_i$:
\[
\left[ \begin{array}{ccccc} 1 & x_1 & x_1^2 & \cdots & x_1^n \\
1 & x_2 & x_2^2 & \cdots & x_2^n \\
 && \vdots && \\
 1 & x_n & x_n^2 & \cdots & x_n^n 
 \end{array} \right] \left[ \begin{array}{c} c_0 \\ c_1 \\ c_2 \\ \vdots \\c_n \end{array} \right] = \left[ \begin{array}{c} y_1 \\ y_2 \\ \vdots \\ y_n \end{array} \right]
\]

\item For a given system $\bm{A} \bm{x} = \bm{b}$, the \textit{backslash operator} will solve a system as \texttt{x = A\textbackslash b}. 
\par Given \texttt{x} and \texttt{y} vectors (assume these are already the MATLAB virtual machine's memory), write code to generate the \texttt{A} matrix for the interpolation problem stated above.
\par\textbf{Solution:} 
\begin{lstlisting}[language=Octave]
A = [];
for i = 1:numel(x)
	A = [A x.^i]
end
\end{lstlisting}

%\item \textit{Related linear algebra question:} When measuring physical systems, sometimes the same input may yield slightly different outputs. In this system, $x$ is the input and $y$ is the output, and you can think of the interpolation as trying to determine the response of the system for all $x$. 
%\par Suppose we measure the same $x$ twice and it produces two different $y$ values. (This could be due to measurement error, system noise, or many other causes.) Will we still be able to find a unique interpolating polynomial by setting up this matrix equation? 
%\par \textbf{Solution:} 

\end{enumerate}

\end{enumerate}



\end{document}