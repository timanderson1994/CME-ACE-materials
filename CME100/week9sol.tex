\documentclass[letterpaper, 11pt]{article}
\usepackage{comment} % enables the use of multi-line comments (\ifx \fi) 
\usepackage{lipsum} %This package just generates Lorem Ipsum filler text. 
\usepackage{fullpage} % changes the margin

\usepackage{fancyhdr} % Required for custom headers
\usepackage{lastpage} % Required to determine the last page for the footer
\usepackage{extramarks} % Required for headers and footers
\usepackage{mdframed}
\usepackage{caption}
\usepackage{subcaption}
\usepackage{float}
\usepackage{array}
\usepackage{soul}
\usepackage{amsmath}
\usepackage{graphicx} % Required to insert images
\usepackage{multicol}
\usepackage{enumitem}
\usepackage{amssymb,bm}
\usepackage{verbatim,eufrak,hyperref,bbm}
\usepackage{titlesec}

%%%%% TEMPLATE-SPECIFIC FORMATTING %%%%%
%\usepackage{fourier}
\usepackage[adobe-utopia]{mathdesign}
\titleformat{\section}
  {\normalfont\fontsize{12}{15}\bfseries}{\thesection.}{1em}{}
  \titleformat{\subsection}[runin]{\normalfont}{\thesubsection}{3pt}{}
%\usepackage[T1]{fontenc}

%----------------------------------------------------------------------------------------
%	NAME AND CLASS SECTION
%----------------------------------------------------------------------------------------

\newcommand{\hmwkTitle}{Week\ 9\ Solutions} % Assignment title
\newcommand{\hmwkClass}{CME\ 100\ ACE} % Course/class
\newcommand{\hmwkAuthorName}{Timothy Anderson} % Your name
\newcommand{\hmwkAuthorEmail}{timmya@stanford.edu} % Your email

% Set up the header and footer
\pagestyle{fancy}
\lhead{} % Top left header
\chead{} % Top center header
\rhead{} % Top right header
\lfoot{\hmwkClass\ : \hmwkTitle} % Bottom left footer
\cfoot{Page\ \thepage\ of\ \pageref{LastPage}} % Bottom center footer
\rfoot{\hmwkAuthorName} % Bottom right footer
\renewcommand\headrulewidth{0pt} % Size of the header rule
\renewcommand\footrulewidth{0.4pt} % Size of the footer rule


% Math commands
\DeclareMathOperator*{\argmin}{arg\,min}
\DeclareMathOperator*{\argmax}{arg\,max}
\allowdisplaybreaks

% Margins
\topmargin=-0.45in
\evensidemargin=0in
\oddsidemargin=0in
\textwidth=6.5in
\textheight=9.0in
\headsep=0.25in 

\setlength{\parindent}{0pt} % Set indent to zero

\begin{document}

%\thispagestyle{empty}
\noindent
\normalsize 
%\hmwkAuthorName 
\hmwkClass \hfill May\ 29,\ 2017\\
%\hmwkAuthorEmail \\

\begin{center} \Large \textbf{\hmwkTitle} \end{center}

\section{Work}
% TC 16.2 #27
\subsection{} Find the work done by the force $\bm{F} = xy\bm{i}+(y-x)\bm{j}$ over the straight line from $(1,1)$ to $(2, 3)$.
\par \textbf{Solution:} The first step when solving a problem such as this is to find the parameterized curve over which we are integrating. Here, it is a straight line, so we can set up the curve from $P_1$ to $P_2$ as $\bm{r}(t) = (1-t)P_1 + tP_2 $ and take $t \in [0,1]$. So, our parameterized path is:
\[ \bm{r}(t) = (1 -t)(\bm{i} + \bm{j}) + t(2\bm{i} + 3\bm{j}) = (1+t)\bm{i} + (1 + 2t)\bm{j} \]
Now, recall the formula for work:
\[ W = \int_C \bm{F} \cdot \bm{T} dx = \int_a^b \bm{F}(\bm{r}(t)) \cdot \frac{ d\bm{r}}{dt}dt \]
Solving the problem is now just a matter of computation:
\begin{gather*}
 W =  \int_a^b \bm{F}(\bm{r}(t)) \cdot \frac{ d\bm{r}}{dt}dt = \int_0^1 \left[ (1+t)(1 + 2t) + 2(1+2t - 1 - t)\right]dt \\
= \int_0^1 \left[ 1 + 5t + 2t^2  \right] dt = \left[ t + \frac{5}{2} t^2 + \frac{2}{3} t^3 \right]_0^1 = \frac{25}{6} \quad\blacksquare
\end{gather*}

\subsection{} Find the work done by the gradient of \[f(x, y) = (x + y)^2\] counterclockwise around the circle $x^2 + y^2 = 4$ from $(2, 0)$ to itself.
\par \textbf{Solution:} In CME 100, if a problem seems incredibly hard, your first reflex should be that there must be a simple but perhaps unintuitive way to solve the problem. In the case of work integrals, the main ``trick'' we have available is conservative fields. Recall that a vector field $\bm{F}$ is conservative if and only if there exists some function $f$ such that
\[ \nabla f = \bm{F}\]
In this problem, we explicitly define the field $\bm{F}$ as the gradient of a given function, and the work is computed around a closed loop, so trivially we have
\[ W = 0 \quad \blacksquare \]

% TC 16.2 #30
\section{Flux \& Circulation}
Find the flux of the fields:
\[ \bm{F}_1 = 2x\bm{i} - 3y\bm{j}, \qquad \bm{F}_2 = 2x\bm{i} + (x - y)\bm{j} \]
about the path:
\[r(t) = a\cos(t)\bm{i} + a\sin(t)\bm{j},\; 0 \leq t \leq 2 \pi \]
\par \textbf{Solution:} For the first field, we have $M = 2x$ and $N = -3y$. Recall that we can rewrite the flux equation as:
\[ \text{Flux} = \int_{t_1}^{t_2} \left(M(x,y) \frac{dy}{dt} - N(x,y) \frac{dx}{dt}\right) dt \]
The velocity vector is:
\[ \frac{ d\bm{r}}{dt} = - a\sin(t) \bm{i} + a \cos (t) \bm{j} \]
Using this, we can easily calculate flux using the parametric equations:
\begin{align*}
\text{Flux} &= \int_{t_1}^{t_2} \left(M(x,y) \frac{dy}{dt} - N(x,y) \frac{dx}{dt}\right) dt \\
&= \int_0^{2\pi} \left( 2(a\cos(t))(a \cos(t))  -3(a\sin(t))(a \sin(t))  \right) dt \\
&= 2a^2 \int_0^{2\pi} \cos^2(t)dt  -3a^2\int_0^{2\pi} \sin^2(t)dt \\
&= 2a^2 \left[ \frac{t}{2}+ \frac{ \sin (2t)}{4} \right]_0^{2\pi} - 3a^2 \left[ \frac{t}{2} - \frac{ \sin(2t)}{4} \right]_0^{2\pi}\\
&= -\pi a^2 \quad\blacksquare
\end{align*}

\par For the second field, we have:
\begin{align*}
\text{Flux} &= \int_{t_1}^{t_2} \left(M(x,y) \frac{dy}{dt} - N(x,y) \frac{dx}{dt}\right) dt \\
&=  \int_{0}^{2 \pi} \left(2(a\cos(t)) (a \cos(t)) - (a \cos(t) - a \sin(t))(-a\sin(t))\right) dt \\
&=  \int_{0}^{2 \pi} \left(2a^2 \cos^2(t) + a^2 \cos(t)\sin(t) - a^2 \sin^2(t)\right) dt \\
&=  2a^2 \int_{0}^{2 \pi} \cos^2(t)dt + a^2 \int_{0}^{2 \pi}\cos(t)\sin(t)dt - a^2 \int_{0}^{2 \pi}\sin^2(t)dt \\
&=  2a^2 \left[ \frac{t}{2} + \frac{ \sin (2t)}{4}  \right]_0^{2\pi} + a^2\left[ \frac{1}{2} \sin^2(t) \right]_0^{2\pi} - a^2 \left[ \frac{t}{2} - \frac{ \sin (2t)}{4}  \right]_0^{2\pi} \\
&= \pi a^2 \quad\blacksquare
\end{align*}


\section{Conservative Fields \& Potential Functions}
\subsection{} Determine which of the following fields are conservative fields. 
\begin{enumerate}[label=(\alph*)]
% TC 16.3 #2
\item $\bm{F} = y \sin(z)\bm{i} + x \sin(z)\bm{j} + xy \cos(z)\bm{k} $
\par \textbf{Solution:} A conservative field always have an associated potential function such that $\nabla f = \bm{F}$. So, for our vector field
\[ \bm{F} = M \bm{i} + N \bm{j} + P \bm{k}\]
we must have 
\[\frac{ \partial P}{\partial y} = \frac{ \partial N}{\partial z}, \quad \frac{ \partial M}{\partial z} = \frac{ \partial P}{\partial x}, \quad \frac{ \partial N}{\partial x} = \frac{ \partial M}{\partial y}\]
because this is the same as saying
\[ \frac{\partial^2 f}{\partial y \partial z} = \frac{\partial^2 f}{\partial z \partial y}, \quad \frac{\partial^2 f}{\partial z \partial x} = \frac{\partial^2 f}{\partial x \partial z}, \quad\frac{\partial^2 f}{\partial x \partial y} = \frac{\partial^2 f}{\partial y \partial x} \]
So, we can apply this test to determine if the field is conservative:
\begin{gather*}
\frac{ \partial (xy\cos(z))}{\partial y} = \frac{ \partial x \sin(z)}{\partial z} \implies x \cos(z) = x \cos(z) \\
 \frac{ \partial y \sin(z)}{\partial z} = \frac{ \partial (xy\cos(z))}{\partial x} \implies y \cos(z) = y \cos (z)  \\
  \frac{ \partial x \sin(z)}{\partial x} = \frac{ \partial y \sin(z)}{\partial y} \implies \sin(z) = \sin(z)
\end{gather*}
So, the field is conservative

\item $\bm{F}=-y\bm{i}+x\bm{j}$
\par \textbf{Solution:} We only have two components, so the conditions for a conservative field reduces to:
\[ \frac{ \partial N}{\partial x} = \frac{ \partial M}{\partial y}\]
Using this:
\[ \frac{\partial (x)}{\partial x} = \frac{\partial (-y)}{\partial y} \implies 1 \neq -1\]
So, the field is not conservative. 


\item $\bm{F} = e^x\cos(y)\bm{i} - e^x\sin(y)\bm{j} + z\bm{k}$
\par \textbf{Solution:} This has three components, so we must our the original conditions:
\[\frac{ \partial P}{\partial y} = \frac{ \partial N}{\partial z}, \quad \frac{ \partial M}{\partial z} = \frac{ \partial P}{\partial x}, \quad \frac{ \partial N}{\partial x} = \frac{ \partial M}{\partial y}\]
Plugging in the components:
\begin{gather*}
\frac{ \partial (z)}{\partial y} = \frac{ \partial (- e^x\sin(y))}{\partial z} \implies  0 = 0 \\
\frac{ \partial (e^x\cos(y))}{\partial z} = \frac{ \partial (z)}{\partial x} \implies 0 = 0\\
 \frac{ \partial (- e^x\sin(y))}{\partial x} = \frac{ \partial (e^x\cos(y))}{\partial y} \implies - e^x\sin(y) = - e^x \sin(y) 
\end{gather*}
So, the field is conservative. 
\par \textit{For extra practice:} try solving for the potential functions (up to an additive constant) of the conservative fields. 

\end{enumerate} 

% TC 16.3 #22
\subsection{} Evaluate the line integral:
\[ \int_{(-1,-1,-1)}^{(2,2,2)} \frac{2x dx + 2y dy + 2z dz}{x^2 + y^2 + z^2} \]
\par \textbf{Solution:} Remember that if we are dealing with a conservative field, we simply need to find the potential function, and find the potential difference to evaluate the line integral. The vector field is just the gradient of the potential function, so to find the potential function we just need to find the function that our gradient came from. To do this, we work component-by-component to find the potential:
\begin{align*}
\frac{\partial f}{\partial x} &= \frac{2x}{x^2 + y^2 + z^2} \implies \ln(x^2 + y^2 + z^2) + g(y,z)\\
\frac{\partial f}{\partial y} &= \frac{2y}{x^2 + y^2 + z^2} = \frac{2y}{x^2 + y^2 + z^2}  + \frac{\partial g}{\partial y} \\
\implies \frac{\partial g}{\partial y} &= 0 \implies g(y,z) = h(z), \implies  f =\ln(x^2 + y^2 + z^2)  + h(z) \\
\frac{\partial f}{\partial z} &= \frac{2z}{x^2 + y^2 + z^2} = \frac{2z}{x^2 + y^2 + z^2} + \frac{\partial h}{\partial z} \\
\implies \frac{\partial h}{\partial z} &= 0 \implies h(z) = C \\
\implies f(x,y,z) &= \ln(x^2 + y^2 + z^2) + C
\end{align*}
So to evaluate the integral, we have:
\[  \int_{(-1,-1,-1)}^{(2,2,2)} \frac{2x dx + 2y dy + 2z dz}{x^2 + y^2 + z^2} = \ln(3(4)) - \ln(3) = \ln(4) \quad\blacksquare \]


% TC 16.3 #18
\subsection{} Evaluate the line integral:
\[ \int_{(0,2,1)}^{(1,\pi/2,2)} 2 \cos(y) dx + \left( \frac{1}{y} - 2x \sin(y) \right) dy  + \frac{1}{z}dz \]
\par \textbf{Solution:} First solve for the potential field:
\begin{align*}
\frac{\partial f}{\partial x} &= 2 \cos(y) \implies f = 2x\cos(y) + g(y,z) \\
\frac{\partial f}{\partial y} &= \left( \frac{1}{y} - 2x \sin (y) \right) = -2x\sin(y) + \frac{\partial g}{\partial y} \\
\implies \frac{\partial g}{\partial y} &= \frac{1}{y} \implies g(y,z) = \ln y + h(z) \\
\frac{\partial f}{\partial z} &= \frac{1}{z} = \frac{\partial h}{\partial z} \implies h(z) = \ln(z) + C\\
\implies f(x,y,z) &=  2x\cos(y) + \ln y + \ln z + C 
\end{align*}
So, we can evaluate the integral as:
\begin{gather*}
 \int_{(0,2,1)}^{(1,\pi/2,2)} 2 \cos(y) dx + \left( \frac{1}{y} - 2x \sin(y) \right) dy  + \frac{1}{z}dz = f(1,\pi/2,2) - f(0,2,1) \\
= 2(1)\cos(\pi/2) + \ln (\pi/2) + \ln 2 - 2(0)\cos(2) - \ln 2 - \ln 1 = \ln (\pi/2)  \quad\blacksquare 
 \end{gather*}


\end{document}

