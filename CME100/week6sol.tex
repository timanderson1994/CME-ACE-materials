\documentclass[letterpaper, 11pt]{article}
\usepackage{comment} % enables the use of multi-line comments (\ifx \fi) 
\usepackage{lipsum} %This package just generates Lorem Ipsum filler text. 
\usepackage{fullpage} % changes the margin

\usepackage{fancyhdr} % Required for custom headers
\usepackage{lastpage} % Required to determine the last page for the footer
\usepackage{extramarks} % Required for headers and footers
\usepackage{mdframed}
\usepackage{caption}
\usepackage{subcaption}
\usepackage{float}
\usepackage{array}
\usepackage{soul}
\usepackage{amsmath}
\usepackage{graphicx} % Required to insert images
\usepackage{multicol}
\usepackage{enumitem}
\usepackage{amssymb,bm}
\usepackage{verbatim,eufrak,hyperref,bbm}
\usepackage{titlesec}

%%%%% TEMPLATE-SPECIFIC FORMATTING %%%%%
%\usepackage{fourier}
\usepackage[adobe-utopia]{mathdesign}
\titleformat{\section}
  {\normalfont\fontsize{12}{15}\bfseries}{\thesection.}{1em}{}
  \titleformat{\subsection}[runin]{\normalfont}{\thesubsection}{3pt}{}
%\usepackage[T1]{fontenc}

%----------------------------------------------------------------------------------------
%	NAME AND CLASS SECTION
%----------------------------------------------------------------------------------------

\newcommand{\hmwkTitle}{Week\ 6\ Solutions} % Assignment title
\newcommand{\hmwkClass}{CME\ 100\ ACE} % Course/class
\newcommand{\hmwkAuthorName}{Timothy Anderson} % Your name
\newcommand{\hmwkAuthorEmail}{timmya@stanford.edu} % Your email

% Set up the header and footer
\pagestyle{fancy}
\lhead{} % Top left header
\chead{} % Top center header
\rhead{} % Top right header
\lfoot{\hmwkClass\ : \hmwkTitle} % Bottom left footer
\cfoot{Page\ \thepage\ of\ \pageref{LastPage}} % Bottom center footer
\rfoot{\hmwkAuthorName} % Bottom right footer
\renewcommand\headrulewidth{0pt} % Size of the header rule
\renewcommand\footrulewidth{0.4pt} % Size of the footer rule


% Math commands
\DeclareMathOperator*{\argmin}{arg\,min}
\DeclareMathOperator*{\argmax}{arg\,max}
\allowdisplaybreaks

% Margins
\topmargin=-0.45in
\evensidemargin=0in
\oddsidemargin=0in
\textwidth=6.5in
\textheight=9.0in
\headsep=0.25in 

\setlength{\parindent}{0pt} % Set indent to zero
\setlength{\parskip}{5.5pt}

\begin{document}

%\thispagestyle{empty}
\noindent
\normalsize 
%\hmwkAuthorName 
\hmwkClass \hfill May\ 8,\ 2017\\
%\hmwkAuthorEmail \\

\begin{center} \Large \textbf{\hmwkTitle} \end{center}

\section{Constrained Optimization}
\subsection{Extrema on a circle} Find the extrema of $f(x,y) = xy$ subject to $x^2 + y^2 = 10$. 
\par \textbf{Solution:} When working through any constrained optimization problem, you need to first state the problem in terms of an \textit{objective} and \textit{constraints}. The objective is the function which we are trying to minimize or maximize, and the constraints are the equations which must also be satisfied. Here, the objective is $f(x,y) = xy$ and the constraint is $x^2 + y^2 = 10$. 
\par Using this we can form the Lagrangian:
\[ \mathcal{L} = xy - \lambda(10 - x^2 - y^2) \]
The Lagrangian actually originates from physics, and forms a very nice analogy with optimization. In physics, the Lagrangian $L = T - U$ is the difference between the kinetic energy $T$ and potential energy $U$, and can be thought of as the excess energy in the system. For optimization, the ``potential energy'' is how much more space we can move around the \textit{feasible set}, and the objective is the kinetic energy. We then want to find a \textit{stationary point} of the Lagrangian, since this will be a point at which the energy in the system is maximized. 
\par From here, we take the partial derivatives with respect to each of the variables. (Before proceeding with this step, be sure you are clear about which quantities are constants and which are variables in the problem.)
\begin{align*}
\frac{\partial \mathcal{L}}{\partial x} &= y + 2\lambda x = 0\\
\frac{\partial \mathcal{L}}{\partial y} &= x + 2 \lambda y = 0
\end{align*}
Combining these with the constraint $10 = x^2 + y^2$, we can solve for the optimal quantities $x$, $y$, and $\lambda$:
\begin{gather*}
x^2 + y^2 = 10 \\
y + 2\lambda x = 0 \implies  y^2 =  4 \lambda^2 x^2 \\
 x + 2 \lambda y = 0 \implies x^2 = 4\lambda^2 y^2 \\
10 =  \implies x^2 + y^2 = 4\lambda^2 y^2 + 4 \lambda^2 x^2 = 4\lambda^2 (y^2 + x^2)\\
1 = 4 \lambda^2 \implies \lambda = \pm \frac{1}{2} \quad\blacksquare
\end{gather*}
For the first case $\lambda = 1/2$:
\begin{gather*}
y + 2(1/2) x = 0 \implies y = -x\\
x^2 + (-x)^2 = 10 \implies x = \pm \sqrt{5},\; y = \mp \sqrt{5} \quad\blacksquare
\end{gather*}
For the second case $\lambda = -1/2$:
\begin{gather*}
y + 2(-1/2) x = 0 \implies y = x\\
x^2 + (x)^2 = 10 \implies x = \pm \sqrt{5},\; y = \pm \sqrt{5} \quad\blacksquare
\end{gather*}
So, there are minima at $(\pm \sqrt{5}, \mp \sqrt{5})$ and maxima at $(\pm \sqrt{5}, \pm \sqrt{5})$. 

\par \textit{Note:} the idea of Lagrange multipliers is very closely related to the \textit{principle of least action} in physics, which is one of (if not the most important) concept in physics. While not directly related to CME 100, it is definitely worth reading up on. Richard Feynman gives an excellent exposition in his lectures on physics. 

\subsection{Constrained minimum} Find the points on the curve $xy^2 = 2$ nearest the origin.
\par \textbf{Solution:} This problem first requires you to determine what the objective is. The question asks for the point nearest the origin, which from the distance formula we can write:
\[ D = \sqrt{ (x - 0)^2 + (y - 0)^2} = \sqrt{x^2 + y^2} \]
Now that we know the objective, it is very straight forward to solve using the method of Lagrange multipliers:
\begin{gather*}
\mathcal{L} = \sqrt{x^2 + y^2} - \lambda(2 - xy^2) \\
\frac{\partial \mathcal{L}}{\partial x} = \frac{2x}{2\sqrt{x^2 + y^2}}+ \lambda y^2 = 0\\
\frac{\partial \mathcal{L}}{\partial y} = \frac{ 2y}{2\sqrt{ x^2 + y^2}} + 2\lambda xy = 0\\
\intertext{Rearranging these:}
 \frac{x}{\sqrt{x^2 + y^2}} = - \lambda y^2 \\
 \frac{ 1}{\sqrt{ x^2 + y^2}} = - 2\lambda x
\intertext{Dividing the equations}
x = \frac{y^2}{2x} \implies x^3 = \frac{xy^2}{2} \implies x^3 = 1 \implies x= 1 \quad\blacksquare\\
\intertext{Solving for $y$ with the constraint:}
(1) y^2 = 2 \implies y = \pm \sqrt{2} \quad\blacksquare
\end{gather*}
So, we achieve maximums at $(1,\sqrt{2})$ and $(1, - \sqrt{2})$. 
\par \textit{Note 1:} this is one of the rare instances where we did not need to solve out for the multiplier to find the final answer. This is a lucky accident of this problem's structure---in general, you will need to solve out for the value of the multiplier. (Indeed, if you take more advanced optimization courses, you will see that solving for the multiplier is almost as important as solving for the original variables.)
\par \textit{Note 2:} we can express the distance $D = f(x,y) = f(h(x,y))$ where $f(\cdot)$ is an invertible increasing function. Because $f(\cdot)$ is invertible and increasing, we actually could of discarded the outer function and optimized $h(x,y)$. In this problem, this would be optimization $x^2 + y^2$ instead of $\sqrt{x^2 + y^2}$. The answer is the same, but the math is noticeably simpler when we use the alternative objective. This problem is a good example of how exploiting a problem's structure can make your life easier in solving optimization problems. 

\subsection{Maximizing a product} Find the largest product the positive numbers $x$, $y$, and $z$ can have if $x^2 + y^2 + z^2 = 16$.
\par \textbf{Solution:} This problem is a bit tricky because it is not entirely clear what the constraints are. The objective is obviously $xyz$ and we have the one constraint $x^2 + y^2 + z^2 = 16$, but we also have the constraints $x,y,z \geq 0$. However, we need not include these in the Lagrangian if we are sure to only consider positive solutions when solving for $x,y,z$. From here, we can form the Lagrangian:
\[ \mathcal{L} = xyz - \lambda(16 - x^2 - y^2 - z^2)  \]
Then, taking partial derivatives with respect to the variables:
\begin{align*}
\frac{\partial \mathcal{L}}{\partial x} &= yz + 2 \lambda x = 0\\
\frac{\partial \mathcal{L}}{\partial y} &= xz + 2 \lambda y = 0\\
\frac{\partial \mathcal{L}}{\partial z} &= xy + 2 \lambda z = 0
\end{align*}
Note the symmetry in the problem. That is, all of the equations are the same with the variables permuted. This immediately tells us that $x = y = z$ when the partials of the Lagrangian are 0. (It is a good exercise to check this by hand using the optimality conditions.)
\par We can then plug this into the constraint to find the variables:
\[ 3x^2 = 16 \implies x = \frac{4}{\sqrt{3}} = y = z \quad\blacksquare \]
So, we reach a maximum at $\left( \frac{4}{\sqrt{3}} ,\frac{4}{\sqrt{3}} ,\frac{4}{\sqrt{3}} \right)$. 
\par \textit{Note:} we once again did not need to solve for the multiplier because we used the problem's symmetry. 

\section{Constrained Variables}
If $f = x^2 + y - z + \sin t$ and $x + y = t$, find:
\begin{enumerate}[label=(\alph*)]
\item $\left( \frac{\partial f}{\partial y}\right)_{x,z}$
\par \textbf{Solution:} in this problem, we have four variables: $f,\; x,\; y,$ and $z$. Working through these types of problems is actually identical to taking partial derivatives as before, and the hardest part is interpreting the notation. You will be asked to compute an expression that looks something like:
\[ \left( \frac{\partial w}{\partial x}\right)_{y,z}\]
This notation says that:
\[ \left( \frac{ \partial [\text{dependent variable}]}{\partial \text{[independent variable]}}\right)_{[\text{other independent variables}]} \]
Now, recall the chain rule for partial derivatives:
\[ \frac{\partial f}{\partial y} = \frac{ \partial f}{\partial x}\frac{\partial x}{\partial y} + \frac{ \partial f}{\partial y}\frac{\partial y}{\partial y} +  \frac{ \partial f}{\partial z}\frac{\partial z}{\partial y}  + \frac{ \partial f}{\partial t}\frac{\partial t}{\partial y}  \]
Finally, we can plug in the remaining derivatives to find our final solution:
\[ \frac{\partial f}{\partial y} = \frac{ \partial f}{\partial x}\frac{\partial x}{\partial y} + \frac{ \partial f}{\partial y}\frac{\partial y}{\partial y} +  \frac{ \partial f}{\partial z}\frac{\partial z}{\partial y}  + \frac{ \partial f}{\partial t}\frac{\partial t}{\partial y}  \]

By definition, 
\[ \frac{ \partial [\text{independent variable \# $i$}]}{ \partial [\text{independent variable \# $j$}]} = \begin{cases} 0 & i \neq j \\ 1 & i = j \end{cases} \]
Essentially, if two variables are independent, then a change in one does not affect the other, or a change in itself trivially causes the same change in itself. Using this, we can rewrite the above chain rule formula based on which variables are independent:
\[ \frac{\partial f}{\partial y} = 1(1) +  \cos(t) (1) =  1 + \cos (t) = 1  + \cos(x + y) \quad\blacksquare\]

\item $\left( \frac{\partial f}{\partial y}\right)_{z,t}$
\par \textbf{Solution:} once again writing the chain rule formula:
\[ \frac{\partial f}{\partial y} = \frac{ \partial f}{\partial x}\frac{\partial x}{\partial y} + \frac{ \partial f}{\partial y}\frac{\partial y}{\partial y} +  \frac{ \partial f}{\partial z}\frac{\partial z}{\partial y}  + \frac{ \partial f}{\partial t}\frac{\partial t}{\partial y}  \]
$z$ and $t$ are independent from $y$, so we can immediately set these partials to 0:
\[ \frac{\partial f}{\partial y} = \frac{ \partial f}{\partial x}\frac{\partial x}{\partial y} + \frac{ \partial f}{\partial y}(1)  \]
$x$ is treated as a dependent variable, so we need to solve for it in terms of the independent variables. Rewriting the constraint equation as:
\[ x = t - y\]
we have
\[ \frac{ \partial x}{\partial y } = -1 \]
Finally, we plug this back into the equation from above to find:
\[ \frac{\partial f}{\partial y} = 2x(-1) + 1(1) = 1 - 2x =1 - 2t + 2y \quad\blacksquare  \]

\item $\left( \frac{\partial f}{\partial z}\right)_{x,y}$
\par \textbf{Solution:} following the same procedure from the previous problem:
\[ \frac{\partial f}{\partial z} = \frac{ \partial f}{\partial x}\frac{\partial x}{\partial z} + \frac{ \partial f}{\partial y}\frac{\partial y}{\partial z} +  \frac{ \partial f}{\partial z}\frac{\partial z}{\partial z}  + \frac{ \partial f}{\partial t}\frac{\partial t}{\partial z}  \]
$x$ and $y$ are independent from $z$, so:
\[ \frac{\partial f}{\partial z} =  \frac{ \partial f}{\partial z}  + \frac{ \partial f}{\partial t}\frac{\partial t}{\partial z}  \]
However, $t$ by definition is independent from $z$, so we can simply compute the answer as 
\[ \frac{ \partial f }{\partial z} = -1 \quad\blacksquare \]
\end{enumerate}

\section{Multi-dimensional Integrals}
\subsection{} Compute the following iterated integrals. 
\begin{enumerate}[label=(\alph*)]
\item $\int_0^1 \int_1^2 x y e^x dy dx $
\par \textbf{Solution:} the first integral is very straightforward to compute:
\[ \int_0^1 \int_1^2 x y e^x dy dx = \int_0^1\left[ \frac{1}{2}y^2\right]_1^2 x e^x dx = \frac{3}{2} \int_0^1 xe^x dx \]
This second integral is a bit tricky, but you should definitely be comfortable computing integrals such as this. CME 100 is essentially a class in computing different quantities and does not emphasize any theory, so you should be very comfortable computing integrals in closed form. 
\par To compute this integral, you should use the \textit{tabular method} of integration by parts. If you are not familiar with this method, please look it up and quickly familiarize yourself with it. It is by far the easiest and most straight forward way of doing integration by parts, and is essential to know for CME 100. Using the tabular method:
\begin{align*}
\mathbf{d\cdot/dx} &\qquad \mathbf{\int \cdot dx} \\
x & \qquad e^x \\
1 & \qquad e^x \\
0 & \qquad e^x 
\end{align*}
So the result is
\[  \int_0^1 xe^x dx = \left[ xe^x - e^x \right]_0^1 = e - e + 1\]
So the final answer is:
\[  \int_0^1 \int_1^2 x y e^x dy dx = \frac{3}{2}(1) = \frac{3}{2} \quad\blacksquare \]

\item $\int_\pi^{2 \pi} \int_0^\pi (\sin x + \cos y) dx dy$
\par \textbf{Solution:} there is a slight trick in this function. While it is tempting to integrate the $x$ and $y$ terms separately, this is actually not quite the case. Rewrite the integral as
\[ \int_\pi^{2 \pi} \int_0^\pi (\sin x + \cos y) dx dy = \int_\pi^{2 \pi} \int_0^\pi \sin x dx dy + \int_\pi^{2 \pi} \int_0^\pi  \cos y dx dy \]
Now, we should do the integrals separately so we are sure not to miss any terms:
\begin{gather*}
\int_\pi^{2 \pi} \int_0^\pi \sin x dx dy + \int_\pi^{2 \pi} \int_0^\pi  \cos y dx dy  \\
\int_\pi^{2 \pi} \left[ - \cos x \right]_0^\pi dy + \int_\pi^{2 \pi} \pi  \cos y dy  \\
2 \int_\pi^{2 \pi} dy + \pi \int_\pi^{2 \pi} \cos y dy  \\
2 \pi + 0 = 2 \pi \quad \blacksquare 
\end{gather*}

\item $\int_0^4 \int_1^2 \frac{\sqrt{x}}{y^2} dy dx$
\par \textbf{Solution:}
\[
\int_0^4 \int_1^2 \frac{\sqrt{x}}{y^2} dy dx = 
\int_0^4  \sqrt{x} \left[ -\frac{1}{y}\right]_1^2 dy dx = 
\frac{1}{2} \int_0^4 \sqrt{x} dx = 
\frac{1}{2} \left[\frac{2}{3} x^{3/2} \right]_0^4 = \frac{1}{3}(8 - 0) = \frac{8}{3} \quad\blacksquare
\]

\end{enumerate}

% TC 15.1 #24
\subsection{} Find the volume of the region bounded above by the ellipitical paraboloid $z = 16 - x^2 - y^2$ and below by the square $R: 0 \leq x \leq 2,\; 0 \leq y \leq 2$.
\par \textbf{Solution:} the only trick here is setting up the bounds of integration to do the volume integral. From there, it is just a matter of evaluating the iterated integral to find the volume
\begin{gather*}
\int_0^2 \int_0^2 \int_0^{16 - x^2 - y^2} dz  dy dx = \int_0^2 \int_0^2 (16 - x^2 - y^2)  dy dx = \int_0^2 \left[16y - yx^2 - \frac{1}{3}y^3  \right]_0^2 dx \\
\int_0^2 \left(32 - 2x^2 - \frac{8}{3}\right) dx = \int_0^2 \left(\frac{88}{3} - 2x^2\right)dx = \left[ \frac{88}{3}x - \frac{2}{3}x^3 \right]_0^2 = \frac{160}{3} \quad\blacksquare
\end{gather*}

% TC 15.2 #42
\subsection{} Evaluate by reversing the order of integration: $\int_0^2 \int_{-\sqrt{4 - x^2}}^{\sqrt{4 - x^2}} 6x dy dx$
\par \textbf{Solution:} Notice that the domain is a semicircle in the right half plane. So, if we want the switch the order of integration, we can directly rewrite the bounds of integration. Therefore, $y = [-2,2]$ and $x = [0, \sqrt{4 - y^2}]$ and our integral becomes:
\[ \int_{-2}^2 \int_0^{\sqrt{4 - y^2}} 6x dx dy = \int_{-2}^2 \left[ 3x^2 \right]_0^{\sqrt{4 - y^2}} dy = \int_{-2}^2 3(4 - y^2)dy = \left[ 12y - y^3 \right]_{-2}^2 = 16 +24 - 8 = 32 \quad\blacksquare\]

% TC 15.3 #21
\subsection{} Find the average height of the paraboloid $z =x^2 +y^2$ over the square $0 \leq x \leq 2,\; 0 \leq y \leq 2$.
\par \textbf{Solution:} recall that the average height is the volume divided by the area of the base. The area of the base here is trivially 4, so we need to find the volume:
\begin{gather*}
V = \int_0^2 \int_0^2 \int_0^{x^2 + y^2} dz dy dx = \int_0^2 \int_0^2 (x^2 + y^2) dy dx = \int_0^2 \left[x^2y + \frac{1}{3} y^3 \right]_0^2 dx \\
= \int_0^2 \left(2x^2 + \frac{8}{3} \right) dx = \left[ \frac{2}{3} x^3 + \frac{8}{3}x \right]_0^2 = \frac{32}{3} 
\end{gather*}
So, the average height is 
\[ \frac{32}{3} \frac{1}{4} = \frac{8}{3} \quad\blacksquare \]


\end{document}

