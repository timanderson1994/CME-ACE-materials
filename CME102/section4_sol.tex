%%%%%%%%%%%%%%%%%%%%%%%%%%%%%%%%%%%%%%%%%
% Short Sectioned Assignment
% LaTeX Template
% Version 1.0 (5/5/12)
%
% This template has been downloaded from:
% http://www.LaTeXTemplates.com
%
% Original author:
% Frits Wenneker (http://www.howtotex.com)
%
% License:
% CC BY-NC-SA 3.0 (http://creativecommons.org/licenses/by-nc-sa/3.0/)
%
%%%%%%%%%%%%%%%%%%%%%%%%%%%%%%%%%%%%%%%%%

%----------------------------------------------------------------------------------------
%	PACKAGES AND OTHER DOCUMENT CONFIGURATIONS
%----------------------------------------------------------------------------------------

\documentclass[letterpaper, fontsize=11pt]{scrartcl} % A4 paper and 11pt font size

\usepackage[T1]{fontenc} % Use 8-bit encoding that has 256 glyphs
\usepackage{fourier} % Use the Adobe Utopia font for the document - comment this line to return to the LaTeX default
\usepackage[english]{babel} % English language/hyphenation
\usepackage{amsmath,amsfonts,amsthm} % Math packages

\usepackage{lipsum} % Used for inserting dummy 'Lorem ipsum' text into the template

\usepackage{sectsty} % Allows customizing section commands
\allsectionsfont{\centering \normalfont\scshape} % Make all sections centered, the default font and small caps

\usepackage{fancyhdr} % Custom headers and footers
\usepackage{graphicx}
\usepackage{mcode}
\pagestyle{fancyplain} % Makes all pages in the document conform to the custom headers and footers
\fancyhead{} % No page header - if you want one, create it in the same way as the footers below
\fancyfoot[L]{\textit{CME 102 Spring '15-'16}} % Empty left footer
\fancyfoot[C]{} % Empty center footer
\fancyfoot[R]{Tim Anderson} % Page numbering for right footer
\renewcommand{\headrulewidth}{0pt} % Remove header underlines
\renewcommand{\footrulewidth}{0pt} % Remove footer underlines
\setlength{\headheight}{13.6pt} % Customize the height of the header

\numberwithin{equation}{section} % Number equations within sections (i.e. 1.1, 1.2, 2.1, 2.2 instead of 1, 2, 3, 4)
\numberwithin{figure}{section} % Number figures within sections (i.e. 1.1, 1.2, 2.1, 2.2 instead of 1, 2, 3, 4)
\numberwithin{table}{section} % Number tables within sections (i.e. 1.1, 1.2, 2.1, 2.2 instead of 1, 2, 3, 4)

\setlength\parindent{0pt} % Removes all indentation from paragraphs - comment this line for an assignment with lots of text
\begin{document}

%----------------------------------------------------------------------------------------
%	TITLE SECTION
%----------------------------------------------------------------------------------------

\newcommand{\horrule}[1]{\rule{\linewidth}{#1}} % Create horizontal rule command with 1 argument of height


\section*{Week 4 Section Problems}
\par Solve the following problems. If initial conditions are given, solve for all constants of integration. It is okay to leave answers in implicit form or with unsolved integrals. 

\begin{enumerate}

\item For the following ODEs, give its type (i.e. linear/nonlinear, order, homogeneous/inhomogeneous) and, if possible, solve the ODE.
\begin{enumerate}
\item $\frac{1}{2}y'' + y' + y = 0$ \par
\textbf{Solution} \newline
This is a second order linear constantly coefficients homogeneous ODE. \par 
Solving the characteristic polynomial:
$$\frac{1}{2}\lambda ^2 + \lambda + 1 = 0 $$
$$ \lambda = \frac{-1 \pm \sqrt{1 - 4*1/2}}{2*1/2} =  -1 \pm \imath$$
$$y(x) = e^{-x}(c_1 sin(x) + c_2 cos(x))$$

\item $(y')^2 - y = 0$ \newline
\textbf{Solution} \newline
First-order nonlinear homogeneous ODE. \par
$$ (y')^2 = y$$
$$\frac{dy}{\sqrt{y}} = dx$$
$$2\sqrt{y} = x + c$$
$$y(x) = \frac{1}{4}(x + c)^2$$

\item $xy' + y = x^2$ \par
\textbf{Solution} \newline
First order linear inhomogeneous ODE.
$$y(x) = e^{-\int \frac{1}{x} dx}\bigg(\int e^{\int \frac{1}{x} dx} x dx + c \bigg)$$
$$y(x) = \frac{x^2}{3} + \frac{C}{x}$$
\end{enumerate}
\item \textbf{Existence and uniqueness:} For the following ODE, give the region $R_1$ where the solution exists, and the region $R_2$ where the solution is unique. 
$$ y' + \sqrt{y} ln(x) = 5$$
$$y(5) = 2$$
\textbf{Solution} \newline
For existence, recall the first part of the existence and uniqueness theorem. It states that if there is a region $R_1$ containing $y_0$ within which $f(x,y)$ is continuous, then the solution is unique. For this problem, we have:
$$ y' = f(x,y) = 5 - \sqrt{y}ln(|x|)$$
So, $f(x,y)$ is continuous for $x \in \mathcal{V} = \{x | x \in \mathbb{R}, x \neq 0\}$ and $y \in \mathcal{W} = \{ y | y \geq 0 \}$. We must have $y_0$ within our region $R_1$. $y_0 = (5,2)$, so the region where the function is continuous is then:
$$ R_1 = (x,y) \in \mathcal{Z} = \{x | x > 0 \} \times \{ y | y \geq 0 \}$$
For uniqueness, we need $ \partial f / \partial y $ continuous over some region $R_2 \subseteq R_1$:
$$\frac{\partial f}{\partial y} = \frac{-ln(|x|)}{2\sqrt{y}}$$
which is continuous for $x \neq 0$ and $y > 0$. So we can define 
$$ R_2 = (x,y) \in \mathcal{Z'} = \{x | x > 0 \} \times \{ y | y > 0 \}$$
which is clearly a subset of $R_1$.


\item \textbf{Numerical Stability:} Give the amplification factor and maximum step size for the following numerical methods for the model ODE $y' = \lambda y$, assuming that $\lambda < 0$.
\begin{enumerate}
\item Forward Euler \par
\textbf{Solution} \newline
We can derive the amplification factor beginning with the definition of the numerical method:
$$y_{n+1} = y_n + hy'_n$$
$$y'_n = \lambda y_n$$
$$y_{n+1} = y_n + h\lambda y_n$$
$$y_{n+1} = (1 + h\lambda)y_n = (1 + h \lambda)^{n+1} y_0$$
So we have the amplification factor $\sigma(h) = 1 + h \lambda$. For stability, we need $|\sigma(h)|\leq 1$:
$$\sigma(h) \geq -1$$
$$1 + h\lambda \geq -1 $$
$$h_{max} = \frac{2}{|\lambda|}$$

\item Backward Euler\par
\textbf{Solution} \newline
$$y_{n+1} = y_n + hy'_{n+1} = y_n + h\lambda y_{n+1} $$
$$y_{n+1} = \frac{1}{1 - h\lambda} y_n$$
$$|\sigma(h)| = \bigg|\frac{1}{1 - h\lambda}\bigg| \leq 1 $$
which is valid for all h since $\lambda < 0$, so Backward Euler is \textit{unconditionally stable}.
\end{enumerate}

\item \textbf{ODEs and Eigenvalues:} Transform the following second order ODEs into a system of first order ODEs, solve for the eigenvalues of the resulting matrices, and relate these to the solution to the ODE. 
\begin{enumerate}
\item $y'' - y = 0$ \par
\textbf{Solution} \newline
\par Substitute $v = y'$. This gives the system:
$$\left[ \begin{array}{c}
y' \\
v' \end{array} \right] = 
\left[
\begin{array}{cc} 0 & 1 \\ 1 & 0 \end{array} \right]
\left[ \begin{array}{c}
y \\
v \end{array} \right]$$
\par This matrix has eigenvalues $\lambda_1 = 1$ and $\lambda_2 = -1$. \par Real parts of the eigenvalues correspond to the exponential part of a solution. This ODE has solution $y(x) = c_1 e^x + c_2 e^{-x}$, so the exponential behavior of the system is apparent.

\item $y'' + y = 0$\par

\textbf{Solution} \newline
\par Substitute $v = y'$. 
$$\left[ \begin{array}{c}
y' \\
v' \end{array} \right] = 
\left[
\begin{array}{cc} 0 & 1 \\ -1 & 0 \end{array} \right]
\left[ \begin{array}{c}
y \\
v \end{array} \right]$$
\par This matrix has eigenvalues $\lambda_1 = \imath$ and $\lambda_2 = -\imath$. \par Imaginary terms in the eigenvalue means there will be oscillatory terms in the solution. This follows from Euler's identity ($e^{\imath \pi} + 1 = 0$). This ODE has solution $y(x) = c_1 sin(x) + c_2 cos(x)$, which indeed has oscillatory terms. \par \textbf{Remark:} Consider how this corresponds to stability of a dynamical system. The imaginary part has no effect on stability; only the real part affects stability since this determines whether the system is exponentially growing or decaying in time.


\item $y'' + 2y' + y = 0$\par

\textbf{Solution} \newline
\par Substitute $v = y'$. 
$$\left[ \begin{array}{c}
y' \\
v' \end{array} \right] = 
\left[
\begin{array}{cc} 0 & 1 \\ -1 & -2 \end{array} \right]
\left[ \begin{array}{c}
y \\
v \end{array} \right]$$
\par This matrix has eigenvalue $\lambda = 1$ (a \textit{repeated eigenvalue}). A repeated eigenvalue means that the solution takes the form $y(x) = e^{-x}(c_1 + c_2x)$, which we could also verify by directly solving the homogeneous equation. 

\end{enumerate}

\end{enumerate}

%----------------------------------------------------------------------------------------

\end{document}