%%%%%%%%%%%%%%%%%%%%%%%%%%%%%%%%%%%%%%%%%%%%%%%%%%%%%%%%%%%%%%%%%%%%%%
% writeLaTeX Example: A quick guide to LaTeX
%
% Source: Dave Richeson (divisbyzero.com), Dickinson College
% 
% A one-size-fits-all LaTeX cheat sheet. Kept to two pages, so it 
% can be printed (double-sided) on one piece of paper
% 
% Feel free to distribute this example, but please keep the referral
% to divisbyzero.com
% 
%%%%%%%%%%%%%%%%%%%%%%%%%%%%%%%%%%%%%%%%%%%%%%%%%%%%%%%%%%%%%%%%%%%%%%
% How to use writeLaTeX: 
%
% You edit the source code here on the left, and the preview on the
% right shows you the result within a few seconds.
%
% Bookmark this page and share the URL with your co-authors. They can
% edit at the same time!
%
% You can upload figures, bibliographies, custom classes and
% styles using the files menu.
%
% If you're new to LaTeX, the wikibook is a great place to start:
% http://en.wikibooks.org/wiki/LaTeX
%
%%%%%%%%%%%%%%%%%%%%%%%%%%%%%%%%%%%%%%%%%%%%%%%%%%%%%%%%%%%%%%%%%%%%%%

\documentclass[10pt,landscape]{article}
\usepackage{amssymb,amsmath,amsthm,amsfonts}
\usepackage{multicol,multirow, graphicx}
\usepackage{calc}
\usepackage{ifthen}
\usepackage[landscape]{geometry}
\usepackage[colorlinks=true,citecolor=blue,linkcolor=blue]{hyperref}
\usepackage{enumitem}
\usepackage{parskip, bm}


\ifthenelse{\lengthtest { \paperwidth = 11in}}
    { \geometry{top=.25in,left=.25in,right=.25in,bottom=.25in} }
	{\ifthenelse{ \lengthtest{ \paperwidth = 297mm}}
		{\geometry{top=1cm,left=1cm,right=1cm,bottom=1cm} }
		{\geometry{top=1cm,left=1cm,right=1cm,bottom=1cm} }
	}
\pagestyle{empty}
\makeatletter
\renewcommand{\section}{\@startsection{section}{1}{0mm}%
                                {-1ex}% plus -.5ex minus -.2ex}%
                                {0.5ex}% plus .2ex}%x
                                {\normalfont\large\bfseries}}
\renewcommand{\subsection}{\@startsection{subsection}{2}{0mm}%
                                {-1ex}% plus -.5ex minus -.2ex}%
                                {0.5ex}% plus .2ex}%
                                {\normalfont\normalsize\bfseries}}
\renewcommand{\subsubsection}{\@startsection{subsubsection}{3}{0mm}%
                                {-1ex}% plus -.5ex minus -.2ex}%
                                {1ex}% plus .2ex}%
                                {\normalfont\small\bfseries}}
\makeatother
\setcounter{secnumdepth}{0}
\setlength{\parskip}{3pt}
\setlength{\parindent}{-0.2in}
\setlength{\leftskip}{0.2in}


\setlist[itemize]{noitemsep, topsep=0pt, leftmargin= 0.5cm}
\setlist[enumerate]{noitemsep, topsep=0pt, leftmargin= 0.5cm}


\newcommand{\norm}[1]{\left\lVert#1\right\rVert}
\def\mystrut(#1,#2){\vrule height #1pt depth #2pt width 0pt}

\allowdisplaybreaks[4]
\let \ds \displaystyle
\makeatletter
\newcommand*{\rom}[1]{\expandafter\@slowromancap\romannumeral #1@}
\makeatother



% -----------------------------------------------------------------------

\begin{document}

\raggedright
\footnotesize

\begin{center}
     \Large{\textbf{CME 102 ACE -- Midterm 1 Reference Sheet}} \\
\end{center}
\begin{multicols}{3}
\setlength{\premulticols}{1pt}
\setlength{\postmulticols}{1pt}
\setlength{\multicolsep}{1pt}
\setlength{\columnsep}{2pt}

\section{First-Order ODE}
\subsection{Separation of Variables}
For any nonlinear first order ODE, manipulate to be in form $f(y)dy = g(x)dx$ then integrate.

\par Two special cases for substitution:
\begin{itemize}
\item ODE of form $y' = f(y/x)$, use $u = y/x$
\item ODE of form $y' = f(ay + bx + c)$, use $u = ay + bx + c$
\end{itemize}

\subsection{Linear Inhomogeneous}
ODEs of form $y' + p(x)y = r(x)$
\par Closed form solution:
\[ y(x) = e^{-\int p(x) dx} \left[ \int e^{\int p(x) dx} r(x)dx + C\right]\]
\textbf{Bernoulli Equation:} $y' + p(x) y = q(x)y^n$
\par Solve by substituting $u = y^{1-n}$ to find ODE:
\[u' + (1-n)p(x)u = (1-n)q(x)\]
and solve for $u(x)$ to find $y(x)$

\section{Equilibrium Solutions}

\par System must be \textbf{autonomous} to have equilibria. 
\par To find equilibrium solutions of $y' = f(y)$:
\begin{enumerate}
\item Find zeros of $f(y)$
\item Pick points in-between/outside of the zeros
\item Calculate $y'$ at the test points
\item Classify based on sign of $y'$ between points
\end{enumerate}

%\section{Circuits}
%Kirchoff's voltage law: $\sum_i V_i = 0$ i.e. sum of all voltage drops across elements is zero (conservation of energy)
%\par To form circuit equations:
%\begin{enumerate}
%\item Apply Kirchoff's voltage law around loop (or loops)
%\begin{itemize}
%\item If there are multiple voltage loops i.e. you are forming a system of ODEs, make sure loops do not overlap
%\end{itemize}
%\item If there is a capacitor (and therefore $\int_0^t i(\tau)d\tau$ appears in the equation), take derivation through equation to form second order ODE
%\item Solve second order linear ODE using Laplace transform or other method
%\end{enumerate}
%
%Voltage drops for elements:
%\begin{itemize}
%\item Inductor: $v = Li'$
%\item Resistor: $v = Ri$ (Ohm's law)
%\item Capactor: $v = \frac{Q}{C} = \frac{\int_0^t i(\tau)d\tau}{C}$ (charge over capacitance)
%\end{itemize}


\section{Numerical Methods for IVP's}

\subsection{Accuracy}
\begin{itemize}
\item Local error: error incurred over one step
\item Global error: total error over the domain, one order of $h$ less than local error, calculated as $\epsilon_{global} = N \times \epsilon_{local}$
\end{itemize}

\subsection{Stability}
\begin{itemize}
\item Derive amplification factor $\sigma(h)$ by starting with the model equation $y' = \lambda y$ and ``stepping-through'' the numerical method to derive relationship:
\[ y_{n+1} = \sigma(h) y_n\]
\item Stability condition is $|\sigma(h)| < 1$
\item To find stable $h$, solve $\sigma(h) < 1$ and $\sigma(h) > -1$
\end{itemize}

\subsection{Euler}
\textbf{Forward Euler:} explicit, $\mathcal{O}(h)$ global accuracy:
\[ y_{n+1} = y_n + h f(x_n, y_n) \]
\textbf{Backward Euler:} implicit, $\mathcal{O}(h)$ global accuracy:
\[ y_{n+1} = y_n + h f(x_{n+1}, y_{n+1}) \]
When writing code for implicit scheme, need to first algebraically solve for $y_{n+1}$ in terms of $x_{n+1}$, $y_n$, and $h$.

\section{Trigonometric Identities}
\par \textbf{Regular trigonometric identities:}
\begin{gather*}
\sin^2 x + \cos^2 x = 1, \; \tan^2 x + 1 = \sec^2 x, 1 + \cot^2 x = \csc^2 x \\
\sin (2x) = 2 \sin x \cos x, \; \tan(2x) = \frac{ 2 \tan x}{1 - \tan^2 x}\\
\cos (2x) = \cos^2 x - \sin^2 x = 2 \cos^2 x - 1 = 1 - 2 \sin^2 x
\end{gather*}

\par \textbf{Hyperbolic trigonometric functions:}
\begin{gather*}
\sinh x = \frac{ e^x - e^{-x}}{2}, \; \cosh x = \frac{ e^x + e^{-x}}{2}, \quad \tanh x  = \frac{e^x - e^{-x}}{e^x + e^{-x}} \\
\cosh^2 x - \sinh^2 x = 1, \; \tanh^2 x + \text{sech}^2 x = 1, \; \coth^2 x - \text{csch}^2 x = 1 \\
\sinh(2x) = 2 \sinh x \cosh x, \quad \cosh(2x) = 2 \cosh^2 x - 1\\
\tanh(2x) = \frac{ 2 \tanh x}{1 + \tanh^2 x}
\end{gather*}

\section{Useful Integrals/Derivatives}
\par \textbf{Trigonometric function derivatives:}
\begin{gather*}
\frac{d}{dx} \sin x = \cos x, \; \frac{d}{dx} \cot x = - \csc^2 x , \; \frac{d}{dx}\arcsin x = \frac{1}{\sqrt{1 - x^2}}\\
\frac{d}{dx}\cot x = - \sin x , \;\frac{d}{dx}\sec x = \sec x \tan x, \; \frac{d}{dx}\arccos x = \frac{ -1}{\sqrt{1 - x^2}}\\
\frac{d}{dx} \tan x = \sec^2 x , \; \frac{d}{dx} \csc x = - \csc x \cot x, \; \frac{d}{dx}\arctan x = \frac{1}{x^2 +1}
\end{gather*}

\par \textbf{Trigonometric and other integrals:}
\begin{gather*}
\int \tan x dx = - \log|\cos x| + C, \quad \int \cot x dx = \log | \sin x | + C \\
\int \csc x dx = - \log |\csc x + \cot x| + C \\
\int \sec x = \log |\sec x + \tan x| + C\\
\int \sin^2(x) dx = \frac{1}{2}x - \frac{1}{4} \sin(2x) + C \\
\int \cos^2(x)dx = \frac{1}{2} x + \frac{1}{4} \sin(2x) + C, \; \int \ln x dx = x \ln x - x + C \\
\int e^{ax} \sin (bx)dx = \frac{e^{ax}}{a^2 + b^2}(a \sin bx - b \cos bx) + C \\
\int e^{ax} \cos (bx)dx = \frac{e^{ax}}{a^2 + b^2}(a \cos bx + b \sin bx) + C 
\end{gather*}

\par \textbf{Inverse trigonometric function integrals:}
\begin{gather*}
\int \frac{dx}{x^2 + a^2} = \frac{1}{a} \arctan \frac{x}{a} + C, \; \int \frac{dx}{\sqrt{a^2 - x^2}} = \arcsin \frac{x}{a} + C 
\end{gather*}

\par \textbf{Hyperbolic trig function derivatives:}
\begin{gather*}
\frac{d}{dx} \sinh (x) = \cosh (x), \quad \frac{d}{dx}\cosh (x) = \sinh (x)\\
\frac{d}{dx}\tanh x = 1 - \tanh^2 (x), \quad \frac{d}{dx}\text{csch} (x) = - \text{coth}(x) \; \text{csch}(x) \\
\frac{d}{dx}\text{sech} (x) = - \tanh x \; \text{sech} (x), \quad
\frac{d}{dx}\coth x = 1 - \coth^2 (x)
\end{gather*}







%\section{Delimiters}
%\begin{tabular}{lll}
%\emph{description} & \emph{command} & \emph{output}\\
%parentheses &\verb!(x)! & (x)\\
%brackets &\verb![x]! & [x]\\
%curly braces& \verb!\{x\}! & \{x\}\\
%\end{tabular}


\end{multicols}

\end{document}
