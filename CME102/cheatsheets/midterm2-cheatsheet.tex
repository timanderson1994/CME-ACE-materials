%%%%%%%%%%%%%%%%%%%%%%%%%%%%%%%%%%%%%%%%%%%%%%%%%%%%%%%%%%%%%%%%%%%%%%
% writeLaTeX Example: A quick guide to LaTeX
%
% Source: Dave Richeson (divisbyzero.com), Dickinson College
% 
% A one-size-fits-all LaTeX cheat sheet. Kept to two pages, so it 
% can be printed (double-sided) on one piece of paper
% 
% Feel free to distribute this example, but please keep the referral
% to divisbyzero.com
% 
%%%%%%%%%%%%%%%%%%%%%%%%%%%%%%%%%%%%%%%%%%%%%%%%%%%%%%%%%%%%%%%%%%%%%%
% How to use writeLaTeX: 
%
% You edit the source code here on the left, and the preview on the
% right shows you the result within a few seconds.
%
% Bookmark this page and share the URL with your co-authors. They can
% edit at the same time!
%
% You can upload figures, bibliographies, custom classes and
% styles using the files menu.
%
% If you're new to LaTeX, the wikibook is a great place to start:
% http://en.wikibooks.org/wiki/LaTeX
%
%%%%%%%%%%%%%%%%%%%%%%%%%%%%%%%%%%%%%%%%%%%%%%%%%%%%%%%%%%%%%%%%%%%%%%

\documentclass[10pt,landscape]{article}
\usepackage{amssymb,amsmath,amsthm,amsfonts}
\usepackage{multicol,multirow, graphicx}
\usepackage{calc}
\usepackage{ifthen}
\usepackage[landscape]{geometry}
\usepackage[colorlinks=true,citecolor=blue,linkcolor=blue]{hyperref}
\usepackage{enumitem}
\usepackage{parskip, bm}


\ifthenelse{\lengthtest { \paperwidth = 11in}}
    { \geometry{top=.25in,left=.25in,right=.25in,bottom=.25in} }
	{\ifthenelse{ \lengthtest{ \paperwidth = 297mm}}
		{\geometry{top=1cm,left=1cm,right=1cm,bottom=1cm} }
		{\geometry{top=1cm,left=1cm,right=1cm,bottom=1cm} }
	}
\pagestyle{empty}
\makeatletter
\renewcommand{\section}{\@startsection{section}{1}{0mm}%
                                {-1ex}% plus -.5ex minus -.2ex}%
                                {0.5ex}% plus .2ex}%x
                                {\normalfont\large\bfseries}}
\renewcommand{\subsection}{\@startsection{subsection}{2}{0mm}%
                                {-1ex}% plus -.5ex minus -.2ex}%
                                {0.5ex}% plus .2ex}%
                                {\normalfont\normalsize\bfseries}}
\renewcommand{\subsubsection}{\@startsection{subsubsection}{3}{0mm}%
                                {-1ex}% plus -.5ex minus -.2ex}%
                                {1ex}% plus .2ex}%
                                {\normalfont\small\bfseries}}
\makeatother
\setcounter{secnumdepth}{0}
\setlength{\parskip}{3pt}
\setlength{\parindent}{-0.2in}
\setlength{\leftskip}{0.2in}


\setlist[itemize]{noitemsep, topsep=0pt, leftmargin= 0.5cm}
\setlist[enumerate]{noitemsep, topsep=0pt, leftmargin= 0.5cm}


\newcommand{\norm}[1]{\left\lVert#1\right\rVert}
\def\mystrut(#1,#2){\vrule height #1pt depth #2pt width 0pt}

\allowdisplaybreaks[4]
\let \ds \displaystyle
\makeatletter
\newcommand*{\rom}[1]{\expandafter\@slowromancap\romannumeral #1@}
\makeatother



% -----------------------------------------------------------------------

\begin{document}

\raggedright
\footnotesize

\begin{center}
     \Large{\textbf{CME 102 ACE -- Midterm \#2 Reference Sheet}} \\
\end{center}
\begin{multicols}{3}
\setlength{\premulticols}{1pt}
\setlength{\postmulticols}{1pt}
\setlength{\multicolsep}{1pt}
\setlength{\columnsep}{2pt}

\section{Eigenvalues/Eigenvectors}
For \textit{system} of ODEs of the form $\vec{x}' = \mathbf{A}\vec{x}$:
\begin{enumerate}
\item Assume solution $\vec{x}(t) = C\vec{v} e^{\lambda t}$ and plug in to find $\mathbf{A}\vec{v} = \lambda \vec{v}$

\item Solve for eigenvalues $\lambda$ of $\mathbf{A}$. For a 2x2 matrix $\mathbf{A} = \begin{bmatrix} a & b \\ c & d \end{bmatrix}$:
\[ \lambda = \frac{(a+d) \pm \sqrt{(a+d)^2 - 4(ad-bc)} }{2} \]

\item Solve for eigenvectors of $\mathbf{A}$. Special cases for 2x2 matrix:
\par \textbf{Case 1:} $c \neq 0$:
\[ \vec{v}_1 = \begin{bmatrix} \lambda_1 - d \\ c \end{bmatrix}, \; \vec{v}_2 = \begin{bmatrix} \lambda_2 -d \\ c \end{bmatrix} \]
\par \textbf{Case 2:} $b \neq 0$:
\[ \vec{v}_1 = \begin{bmatrix} b \\  \lambda_1 - a \end{bmatrix}, \; \vec{v}_2 = \begin{bmatrix} b \\ \lambda_2 - a \end{bmatrix} \]
\par \textbf{Case 3:} $b$ and $c$ are zero:
\[ \vec{v}_1 = \begin{bmatrix} 1 \\ 0 \end{bmatrix}, \; \vec{v}_2 = \begin{bmatrix} 0 \\ 1 \end{bmatrix} \]

\item Assemble solution:
\begin{enumerate}
\item Two real, distinct eigenvalues:
\[ \vec{x} = C_1 \vec{v}_1 e^{\lambda_1 t} + C_2 \vec{v}_2 e^{\lambda_2 t} \]

\item One real, repeated eigenvalue: find associated eigenvector $\vec{v}$, solve for $\vec{\rho}$ in 
\[ (\bm{A} - \lambda\bm{I}) \vec{\rho} = \vec{v}\]
then assemble solution:
\[ \vec{x} = C_1 \vec{v} e^{\lambda t} + C_2 \vec{v} t e^{\lambda t} + C_2 \vec{\rho} e^{\lambda t} \]
\item Two complex conjugate eigenvalues $\lambda_{1,2} = \alpha + \beta i$:
\begin{gather*}
\vec{x} = C_1 e^{\alpha t}\left(  \vec{v}_R \cos(\beta t) - \vec{v}_I \sin(\beta t) \right) \\
\qquad + C_2 e^{\alpha t}\left(  \vec{v}_I \cos(\beta t) + \vec{v}_R \sin(\beta t) \right) \end{gather*}

\end{enumerate}

\end{enumerate}

\section{Second-Order Nonlinear}
\subsection{``Missing y'' method}
\par Second order nonlinear ODE that does not contain $y$:
\[ F(y'', y', x) = 0\]
\begin{enumerate}
\item Make substitution $y' = u$ and $y'' = u'$ and rewrite ODE in terms of $u$ and $x$
\item Solve new ODE for $u(x)$
\item Find $y$ as $y = \int u(x) dx$
\end{enumerate}

\subsection{``Missing x'' method}
\par Second order nonlinear ODE that does not contain $x$:
\[ F(y'', y', y) = 0\]
\begin{enumerate}
\item Make substitution $y' = u$ and $y'' = u u'$ and rewrite ODE in terms of $u$ and $y$
\item Solve new ODE for $u(y)$
\item Substitute back $u = y'$ and solve for $y(x)$
\end{enumerate}

\section{Second-Order Linear Homogeneous}
Homogeneous solution has two \textbf{basis functions} $y_1$ and $y_2$. Write homogeneous solution as linear combination of these:
\[ y_h = C_1 y_1 + C_2 y_2 \]

\subsection{Variable Coefficients}
ODE has form:
\[ y'' + p(x) y' + q(x) y = 0\]
Solving using \textbf{reduction of order}. Given $y_1$, find $y_2$ as:
\[ y_2 = y_1 \int \frac{e^{-\int p dx}}{y_1^2} dx \]

\subsection{Constant Coefficients}
ODE has form:
\[ ay'' + by' + cy = 0 \]
with $a,\; b, \; c$ constant. Solve using \textbf{characteristic equation} $y = e^{\lambda x}$. Three cases:
\begin{description}
\item[Case 1] $b^2 - 4ac > 0$ (two distinct real roots):
\begin{gather*}
\lambda_1 = -\frac{b}{2a} + \frac{\sqrt{b^2 - 4ac}}{2a}, \quad \lambda_2 = -\frac{b}{2a} - \frac{\sqrt{b^2 - 4ac}}{2a} \\
y_h = C_1 e^{\lambda_1 x} + C_2 e^{\lambda_2 x}
\end{gather*}
\item[Case 2] $b^2 - 4ac = 0$ (double real root):
\[ \lambda = -\frac{b}{2a}, \quad y_h = (C_1 + C_2x) e^{\lambda x} \]
\item[Case 3] $b^2 - 4ac < 0$ (complex conjugate roots):
\begin{gather*}
\alpha = -\frac{b}{2a}, \quad \beta = \frac{\sqrt{4ac - b^2}}{2a}, \quad \lambda_{1,2} = \alpha \pm i\beta \\
y_h = e^{\alpha x}\left( C_1 \cos(\beta x) + C_2 \sin (\beta x)\right) 
\end{gather*}
\end{description}

\subsection{Euler-Cauchy equation}
ODE has form:
\[ ax^2y'' + bxy' + cy = 0 \]
with $a,\; b, \; c$ constant. Solve using \textbf{characteristic equation} $y = e^{\lambda x}$. Three cases:
\begin{description}
\item[Case 1] $(b-a)^2 - 4ac > 0$:
\begin{gather*}
m_{1,2} = -\frac{b-a}{2a} \pm \frac{\sqrt{(b-a)^2 - 4ac}}{2a} \\
y_h = C_1 x^{m_1} + C_2 x^{m_2}
\end{gather*}
\item[Case 2] $(b-a)^2 - 4ac = 0$:
\[ m = -\frac{b-a}{2a}, \quad y_h = (C_1 + C_2\ln|x|) x^m \]
\item[Case 3] $(b-a)^2 - 4ac < 0$:
\begin{gather*}
\alpha = -\frac{b-a}{2a}, \quad \beta = \frac{\sqrt{4ac - (b-a)^2}}{2a}, \quad m_{1,2} = \alpha \pm i\beta \\
y_h = x^\alpha \left( C_1 \cos(\beta \ln|x|) + C_2 \sin(\beta \ln|x|) \right) 
\end{gather*}
\end{description}


\section{Second-Order Linear Inhomogeneous}
\begin{itemize}
\item You \textbf{must} always solve for the homogeneous solution first
\item The final solution is $y = y_h + y_p$
\item Apply initial conditions \textbf{after} finding the particular solution
\end{itemize}

\subsection{Variation of Parameters}
\begin{itemize}
\item Must use for variable coefficient second order linear ODE
\item Integrals are hard, so only use if undetermined coefficients does not apply
\end{itemize}

\begin{enumerate}
\item Find homogeneous basis solutions $y_1$ and $y_2$
\item Calculate Wronskian:
\[ W = y_1 y_2' - y_2 y_1' \]
\item Put ODE into standard form:
\[ y'' + p(x)y' + q(x)y = r(x) \]
\item Calculate the particular solution:
\[ y_p = -y_1 \int \frac{y_2 r}{W} dx + y_2 \int \frac{y_1 r}{W} dx \]
\end{enumerate}


\subsection{Undetermined Coefficients}
\begin{itemize}
\item Only applicable to constant coefficient equations
\item If you need to solve for more than $\sim4$ constants, variation of parameters probably faster
\end{itemize}

\begin{enumerate}
\item Find homogeneous basis solutions $y_1$ and $y_2$
\item Put ODE into standard form:
\[ y'' + by' + cy = r(x) \]
\item Guess particular solution $y_p(x)$ based on table:
\begin{flalign*}
&\text{Form for $r(x)$:} & \text{Pick $y_p(x)$:} \\
& C \quad &A \\
& x^n & A_n x^n + A_{n-1} x^{n-1} + \cdots + A_1 x + A_0 \\
&e^{\gamma x} & Ae^{\gamma x} \\
& \cos(\omega x) \text{ or } \sin(\omega x) & A\cos(\omega x) + B\sin(\omega x) \\
\end{flalign*}
\begin{itemize}
\item Can have products of these functions (but only these functions) for $r(x)$ and the assumed solution
\item If any part of assumed solution also part of homogeneous basis, apply \textbf{modification rule} by multiplying assumed solution by $x$
\item Sometimes need to use trig identities to make $r(x)$ match something in the table
\end{itemize}
\item Take derivatives of assumed $y_p(x)$ and plug into ODE
\item Algebraically solve for the constants in assumed solution
\end{enumerate}

%\section{Spring-mass Systems}
%All equations start with force balance (Newton's second law):
%\[ \sum_i F_i = mx''\]
%Equation will look like:
%\[ mx''  + \beta x' + kx = 0\]
%\begin{itemize}
%\item $\beta$ is damping coefficient ($\beta = 0$ means undamped motion)
%\item $k$ is spring constant 
%\item $x(t)$ is the \textit{displacement} from the equilibrium position $x_0$
%\end{itemize}
%Rewrite ODE as: $x'' + 2 \mu + \omega^2 x = 0 $. 
%\par \textbf{Undamped motion:} $x(t) = A \sin (\omega t) + B \cos(\omega t)$
%\par \textbf{Damped motion:} when $\beta \neq 0$:
%\begin{description}
%\item[Over Damped:] $\mu > \omega$, 
%\[ x(t) = C_1 e^{(-\mu + \sqrt{\mu^2 - \omega^2})t} + C_2 e^{(-\mu - \sqrt{\mu^2 - \omega^2})t} \]
%\item[Critically Damped:] $\mu = \omega$, shortest decay time
%\[ x(t) = e^{-\mu t}(C_1 + C_2 t) \]
%\item[Underdamped:] $\mu < \omega$, 
%\[ x(t) = e^{-\mu t} \left[ C_1 \cos(\sqrt{\omega^2 - \mu^2} t) + C_2 \sin (\sqrt{\omega^2 - \mu^2}t) \right] \]
%\end{description}


\section{Numerical Methods}

\subsection{Accuracy}
\begin{itemize}
\item Local error: error incurred over one step
\item Global error: total error over the domain, one order of $h$ less than local error, calculated as $\epsilon_{global} = N \times \epsilon_{local}$
\end{itemize}

\subsection{Stability}
\begin{itemize}
\item Derive amplification factor $\sigma(h)$ by starting with the model equation $y' = \lambda y$ and ``stepping-through'' the numerical method to derive relationship:
\[ y_{n+1} = \sigma(h) y_n\]
\item Stability condition is $|\sigma(h)| < 1$
\item To find stable $h$, solve $\sigma(h) < 1$ and $\sigma(h) > -1$
\end{itemize}

\subsection{Euler}
\textbf{Forward Euler:} explicit, $\mathcal{O}(h)$ global accuracy:
\[ y_{n+1} = y_n + h f(x_n, y_n) \]
\textbf{Backward Euler:} implicit, $\mathcal{O}(h)$ global accuracy:
\[ y_{n+1} = y_n + h f(x_{n+1}, y_{n+1}) \]

\subsection{Runge-Kutta Methods}
\textbf{Trapezoidal:} implicit, $\mathcal{O}(h^2)$ global accuracy, related to RK2, average of backward and forward Euler:
\[ y_{n+1} = y_n + \frac{h}{2} \left[ f(x_n, y_n) + f(x_{n+1}, y_{n+1})\right] \]
\textbf{Improved Euler/RK2:} explicit, $\mathcal{O}(h^2)$ global accuracy:
\[ y_{n+1} = y_n + \frac{h}{2} \left[ f(x_n, y_n) + f(x_{n+1}, y_n + h f(x_n,y_n))\right] \]
\textbf{RK4:} explicit, $\mathcal{O}(h^4)$ global accuracy, basis of \texttt{ode45()}:
\begin{gather*}
k_1 = f(t_n, y_n) \\
k_2 = f\left(t_n + \frac{h}{2}, y_n + \frac{h}{2} k_1\right) \\
k_3 = f\left(t_n + \frac{h}{2}, y_n + \frac{h}{2} k_2\right) \\
k_4 = f\left(t_n + h, y_n + hk_3\right) \\
y_{n+1} = y_n + h\left( \frac{1}{6} k_1 + \frac{1}{3} k_2 + \frac{1}{3} k_3 + \frac{1}{6} k_4 \right) 
\end{gather*}


\subsection{Higher-Order Systems of ODEs}
\begin{itemize}
\item ODEs (or systems of ODEs) that contain higher than first derivatives
\item Need to convert to first order to apply numerical methods
\end{itemize}
Create table:
\[ [\text{Old}]\quad |\quad [\text{New}]\quad | \quad [\text{Derivative}] \quad | \quad [\text{New ODE}] \]

\section{MATLAB}
\par \texttt{[t,y] = ode45(@(t,y) myODE(t,y), tspan, y0)}
\par If \texttt{myODE} has two arguments: \texttt{[t,y] = ode45(@myODE, tspan, y0)}
\par Anonymous function: \texttt{f = @(x) [expression]}

\section{Trigonometric Identities}
\par \textbf{Regular trigonometric identities:}
\begin{gather*}
\sin^2 x + \cos^2 x = 1, \; \tan^2 x + 1 = \sec^2 x, 1 + \cot^2 x = \csc^2 x \\
\sin (2x) = 2 \sin x \cos x, \; \tan(2x) = \frac{ 2 \tan x}{1 - \tan^2 x}\\
\cos (2x) = \cos^2 x - \sin^2 x = 2 \cos^2 x - 1 = 1 - 2 \sin^2 x
\end{gather*}

\par \textbf{Hyperbolic trigonometric functions:}
\begin{gather*}
\sinh x = \frac{ e^x - e^{-x}}{2}, \; \cosh x = \frac{ e^x + e^{-x}}{2}, \quad \tanh x  = \frac{e^x - e^{-x}}{e^x + e^{-x}} \\
\cosh^2 x - \sinh^2 x = 1, \; \tanh^2 x + \text{sech}^2 x = 1, \; \coth^2 x - \text{csch}^2 x = 1 \\
\sinh(2x) = 2 \sinh x \cosh x, \quad \cosh(2x) = 2 \cosh^2 x - 1\\
\tanh(2x) = \frac{ 2 \tanh x}{1 + \tanh^2 x}
\end{gather*}

\section{Useful Integrals/Derivatives}
\par \textbf{Trigonometric function derivatives:}
\begin{gather*}
\frac{d}{dx} \sin x = \cos x, \; \frac{d}{dx} \cot x = - \csc^2 x , \; \frac{d}{dx}\arcsin x = \frac{1}{\sqrt{1 - x^2}}\\
\frac{d}{dx}\cos x = - \sin x , \;\frac{d}{dx}\sec x = \sec x \tan x, \; \frac{d}{dx}\arccos x = \frac{ -1}{\sqrt{1 - x^2}}\\
\frac{d}{dx} \tan x = \sec^2 x , \; \frac{d}{dx} \csc x = - \csc x \cot x, \; \frac{d}{dx}\arctan x = \frac{1}{x^2 +1}
\end{gather*}

\par \textbf{Trigonometric and other integrals:}
\begin{gather*}
\int \tan x dx = - \log|\cos x| + C, \quad \int \cot x dx = \log | \sin x | + C \\
\int \csc x dx = - \log |\csc x + \cot x| + C \\
\int \sec x = \log |\sec x + \tan x| + C\\
\int \sin^2(x) dx = \frac{1}{2}x - \frac{1}{4} \sin(2x) + C \\
\int \cos^2(x)dx = \frac{1}{2} x + \frac{1}{4} \sin(2x) + C, \; \int \ln x dx = x \ln x - x + C \\
\int e^{ax} \sin (bx)dx = \frac{e^{ax}}{a^2 + b^2}(a \sin bx - b \cos bx) + C \\
\int e^{ax} \cos (bx)dx = \frac{e^{ax}}{a^2 + b^2}(a \cos bx + b \sin bx) + C 
\end{gather*}

\par \textbf{Inverse trigonometric function integrals:}
\begin{gather*}
\int \frac{dx}{x^2 + a^2} = \frac{1}{a} \arctan \frac{x}{a} + C, \; \int \frac{dx}{\sqrt{a^2 - x^2}} = \arcsin \frac{x}{a} + C 
\end{gather*}

\par \textbf{Hyperbolic trig function derivatives:}
\begin{gather*}
\frac{d}{dx} \sinh (x) = \cosh (x), \quad \frac{d}{dx}\cosh (x) = \sinh (x)\\
\frac{d}{dx}\tanh x = 1 - \tanh^2 (x), \quad \frac{d}{dx}\text{csch} (x) = - \text{coth}(x) \; \text{csch}(x) \\
\frac{d}{dx}\text{sech} (x) = - \tanh x \; \text{sech} (x), \quad
\frac{d}{dx}\coth x = 1 - \coth^2 (x)
\end{gather*}







%\section{Delimiters}
%\begin{tabular}{lll}
%\emph{description} & \emph{command} & \emph{output}\\
%parentheses &\verb!(x)! & (x)\\
%brackets &\verb![x]! & [x]\\
%curly braces& \verb!\{x\}! & \{x\}\\
%\end{tabular}


\end{multicols}

\end{document}
